% */ vim: set tw=150: */
\documentclass[11pt,a4paper]{article}
%\pdfoutput=1

\usepackage{amsmath}
\usepackage[T1]{fontenc}

\usepackage{jheppub}
\usepackage{psfrag}
\usepackage{slashed}
\usepackage{cancel}
\usepackage{lscape}
\usepackage{caption}
\usepackage{array}
\usepackage{graphicx}
\usepackage{subcaption}
\usepackage{multirow}
\usepackage{tabularx}
\usepackage{makecell}
\usepackage[utf8]{inputenc}
\usepackage{amsmath}
\usepackage{amssymb}
\usepackage{relsize}
\usepackage{color}
\usepackage{slashed}
\usepackage{mathtools}
\usepackage{comment}
\usepackage{scalefnt}
\usepackage{siunitx}[=v2]
\sisetup{
    round-mode = off,
    round-precision = 4,
    scientific-notation = false,
    fixed-exponent = 0,
    group-digits = true,
    group-separator = {\,},
}

%{{{ macros

\renewcommand{\arraystretch}{1.3}

\definecolor{mypink}{RGB}{219, 48, 122}
\definecolor{mygreen}{rgb}{0,0.7,0}
\definecolor{raspberry}{rgb}{0.53,0.15,0.34}
\def\SB#1{\textcolor{mygreen}{{\bf\tt [SB: #1]}}}
\def\SZ#1{\textcolor{raspberry}{{\bf\tt [SZ: #1]}}}

%%% spinor products %%%

\def\zz{\boldsymbol{Z}}
\def\mi{\mathrm{MI}}
\def\cF{\mathcal{F}}
\def\cP{\mathcal{P}}
\def\cC{\mathcal{C}}
\def\cA{\mathcal{A}}
\def\cN{\mathcal{N}}
\def\cO{\mathcal{O}}
\def\cD{\mathcal{D}}
\def\cQ{\mathcal{Q}}
\def\cJ{\mathcal{J}}
\def\cI{\mathcal{I}}
\def\cT{\mathcal{T}}
\def\nn{\nonumber \\ }

\def\la{\langle}
\def\ra{\rangle}
\def\spA#1#2{\la#1#2\ra}
\def\spB#1#2{[#1#2]}
\def\spAB#1#2#3{\la#1|#2|#3]}
\def\spBA#1#2#3{[#1|#2|#3\ra}
\def\spAA#1#2#3{\la#1|#2|#3\ra}
\def\spBB#1#2#3{[#1|#2|#3]}
\def\spab#1#2{\la#1|#2]}
\def\spaa#1#2#3#4{\la#1|#2|#3|#4\ra}
\def\spbb#1#2#3#4{[#1|#2|#3|#4]}
\def\wh#1{\widehat#1}
\DeclareMathOperator{\tr}{\rm tr}
\def\trm{\tr_-}
\def\trp{\tr_+}
\def\MP#1#2{(#1\cdot#2)}
\def\trfive{\tr_5}
\def\spAXB#1#2#3#4#5{\la#1|#2|#3|#4|#5]}

\def\bra#1{\langle #1|}
\def\ket#1{|#1 \rangle}
\def\sqbra#1{[#1|}
\def\sqket#1{|#1]}
\def\braket#1{\langle #1 \rangle}

\def\eps{\epsilon}
\def\fl#1{#1^\flat}
\def\tl#1{\tilde{#1}}
\def\wh{\widehat}
\def\tC{\tilde{C}}
\def\qb{{\bar{q}}}
\def\sb{\bar{s}}
\def\Sb{\bar{S}}
\def\lb{\bar{\ell}}
\def\tb{{\bar{t}}}
\def\ttgg{\bar{t}tgg}
\def\mren{\mathrm{mren}}
\def\ren{\mathrm{ren}}
\def\mct{\mathrm{mct}}
\def\ceps{C_\eps}
\def\as{\alpha_s}
\def\dk#1{\frac{d^d k_{#1}}{i\pi^{d/2}e^{-\eps \gamma_E}}}

\def\e{\epsilon}
\def\tT{\tilde{T}}
\def\coll#1#2{\overset{#1||#2}{\to}}
\def\inf{{\rm Inf}}
\def\gg#1{\gamma_{#1}}
\def\XX{\chi}

\def\cv#1#2{\AB{#1}{\gamma^\mu}{#2}}
\def\cvS#1#2#3{\AB{#1}{#2}{#3}}

\def\MHVb{$\overline{\rm MHV}$}
\def\boxX{$\xcancel{\rm\bf box}$}

\def\fl#1{{#1^{\flat}}}
\def\flm#1{{#1^{\flat,\mu}}}
\def\kf#1{{\fl{K_{#1}}}}
\def\kfm#1{{\flm{K_{#1}}}}

\def\ulim#1{\underset{#1}{\lim}}

\def\fbox#1{F^{(#1)}_{\mathrm{box}}}
\def\lh{\hat{L}}
\def\li#1{\mathrm{Li}_{#1}}

\def\finr{{\mathcal{F}}}
\def\pole{{\mathcal{P}}}
\def\cusp{{\mathrm{cusp}}}

\newcommand{\njet}{\texttt{NJet}}
\newcommand{\pentagonfunctions}{\texttt{PentagonFunctions++}}
\newcommand{\finiteflow}{\texttt{FiniteFlow}}
\newcommand{\qd}{\texttt{QD}}
\newcommand{\eigen}{\texttt{Eigen}}
\newcommand{\nnlojet}{\texttt{NNLOJET}}
\newcommand{\mathematica}{\texttt{Mathematica}}
\newcommand{\form}{\texttt{FORM}}
\newcommand{\qgraf}{\texttt{QGRAF}}
\newcommand{\spinney}{\texttt{Spinney}}
\newcommand{\cpp}{\texttt{C++}}
\newcommand*{\NNLOPS}{\textsc{NNLOPS}\xspace}
\DeclareMathOperator{\Tr}{Tr}




%%Christian's command
\newcommand{\eqcite}[1]{(\ref{#1})}
\newcommand{\dd}{\mathop{}\!\mathrm{d}}

\newcommand{\fp}{\texttt{f64}}
\newcommand{\fpp}{\texttt{f128}}
\newcommand{\fppp}{\texttt{f256}}

%%%% typesetting equations %%%%
\def\s#1{s_{#1}}
\def\d#1#2{#1\cdot #2}
\def\p#1{#1}
\def\pp#1{p_{#1}}
\def\f#1{#1^\flat}
\def\n#1{\eta_{#1}}

\def\Adcc{B_n^{(1)}}

\def\usepic#1#2{\parbox{#1}{\includegraphics[width=#1]{#2}}}
\def\usepix#1#2#3#4#5#6{\parbox{#1}{\includegraphics[width=#1,trim= #3 #4 #5 #6,clip=true]{#2}}}

\def\hpl11{{\mathrm{HPL}}_{1,1}}

\newcolumntype{C}[1]{>{\hsize=#1\hsize\centering\arraybackslash}X}%

\newcolumntype{Z}{r<{\hspace{3mm}}}
\newcommand\mc[2]{\multicolumn{1}{>{\centering}p{#2}}{#1}} % handy shortcut macro
% }}}

%%\allowdisplaybreaks

\providecommand{\href}[2]{#2}

\newcommand{\incl}{{\tt inclusive}}
\newcommand{\fidYR}{{\tt fiducial-YR}}
\newcommand{\fidATLAS}{{\tt fiducial-ATLAS}}

\newcommand{\stepone}{{Step\,I}}
\newcommand{\steptwo}{{Step\,II}}
\newcommand{\stepthree}{{Step\,III}}

\newcommand\tS{\tilde{S}}
\newcommand\F{${\textrm F}$}
\newcommand\FJ{${\textrm FJ}$}
\newcommand\FJJ{${\textrm FJJ}$}
\newcommand\PhiBorn{\Phi_{\scriptscriptstyle \textrm B}}
\newcommand\PhiReal{\Phi_{\scriptscriptstyle \textrm R}}
%\newcommand\PhiB{\Phi_{\scriptscriptstyle \textrm H}}
\newcommand\PhiB{\Phi_{\scriptscriptstyle \textrm F}}
\newcommand\PhiBres{\Phi_{\scriptscriptstyle \textrm F,res}}

\newcommand{\flav}{\ell}
\newcommand{\flavBorn}{\flav_{\scriptscriptstyle \textrm B}}
\newcommand{\fullflavBorn}{\hat \flav_{\scriptscriptstyle \textrm B}}
\newcommand{\flavprimeBorn}{\flav'_{\scriptscriptstyle \textrm B}}
\newcommand{\fullflavprimeBorn}{\hat \flav'_{\scriptscriptstyle \textrm B}}
\newcommand{\flavB}{\flav_{\scriptscriptstyle \textrm F}}
\newcommand{\fullflavB}{\hat \flav_{\scriptscriptstyle \textrm F}}
\newcommand{\flavBJ}{\flav_{\scriptscriptstyle \textrm FJ}}
\newcommand{\fullflavBJ}{\hat \flav_{\scriptscriptstyle \textrm FJ}}
\newcommand{\flavBJJ}{\flav_{\scriptscriptstyle \textrm FJJ}}
\newcommand{\fullflavBJJ}{\hat \flav_{\scriptscriptstyle \textrm FJJ}}
\newcommand{\projflav}{\flavB\leftarrow\flavBJ}
\newcommand{\CF}{C_{\mathrm{F}}}
\newcommand{\CA}{C_{\mathrm{A}}}
\newcommand{\NC}{N_{\mathrm{c}}}
\newcommand{\nf}{N_f}
\newcommand{\TF}{T_{\mathrm{F}}}

\newcommand{\flavZg}{\flav_{\scriptscriptstyle Z\gamma}}
\newcommand{\fullflavZg}{\hat \flav_{\scriptscriptstyle Z\gamma}}
\newcommand{\flavZgJ}{\flav_{\scriptscriptstyle Z\gamma {\textrm J}}}
\newcommand{\fullflavZgJ}{\hat \flav_{\scriptscriptstyle Z\gamma {\textrm J}}}
\newcommand{\flavZgJJ}{\flav_{\scriptscriptstyle Z\gamma {\textrm J}}}
\newcommand{\fullflavZgJJ}{\hat \flav_{\scriptscriptstyle Z\gamma {\textrm J}}}
\newcommand{\projflavZg}{\flavZg\leftarrow\flavZgJ}

%\newcommand\PhiBJ{\Phi_{\scriptscriptstyle \textrm HJ}}
\newcommand\PhiBJ{\Phi_{\scriptscriptstyle \textrm FJ}}
\newcommand\ZJ{Z\gamma J}
\newcommand\PhiZJ{\Phi_{\scriptscriptstyle \textrm Z\gamma J}}
\newcommand\PhiBJbar{{\bar \Phi}'_{\scriptscriptstyle \textrm FJ}}
\newcommand\PhiBJJ{\Phi_{\scriptscriptstyle \textrm FJJ}}
\newcommand\PhiZJJ{\Phi_{\scriptscriptstyle \textrm Z\gamma JJ}}
\newcommand\PhiZgam{\Phi_{\scriptscriptstyle \textrm Z\gamma}}
\newcommand{\Fcorr}{F^{\tmop{corr}}_\ell}
\newcommand{\order}[1]{{\cal O}\left(#1\right)}
\newcommand{\aew}{\alpha_{\text{\scalefont{0.77}EW}}} 
\newcommand{\aw}{\alpha_w}
\newcommand{\asCMW}{\alpha_s^{\textrm †CMW}}
%\newcommand{\cO}[1]{{\cal O}\left(#1\right)}
\newcommand{\NNLL}{\text{NNLL}}
\newcommand{\eff}{\epsilon}
\newcommand{\ee}{\ell^+\ell^-}
\newcommand{\kt}[1]{k_{\scaleto{\textrm T}{4pt},#1}}
\newcommand{\veckt}[1]{\vec{k}_{\scaleto{\textrm T}{4pt},#1}}
%\newcommand{\dk}[1]{\langle \mathd k_{#1}\rangle}
\newcommand{\fullF}{\mathcal{F}}
\newcommand{\FNLL}{\mathcal{F}_{\textrm NLL}}
\newcommand{\FNNLL}{\mathcal{F}_{\textrm NNLL}}
\newcommand{\ie}{i.e.\,}
\newcommand{\css}{\text{\scriptsize CSS}}
\newcommand{\zi}{z_i^{(\ell_i)}}
\newcommand{\pt}{p_{\text{\relscale{0.77}T}}}
\newcommand{\GZ}{{\Gamma_Z}}
\newcommand{\GW}{{\Gamma_W}}
\newcommand{\thW}{{\theta_W}}
\newcommand{\mtop}{{m_{\text{\relscale{0.77}top}}}}
\newcommand{\qt}{{q_{\text{\relscale{0.77}T}}}}
\newcommand{\ptarg}[1]{{p_{\text{\relscale{0.77}T,$#1$}}}}
\newcommand{\marg}[1]{{m_{\text{\relscale{0.77}$#1$}}}}
\newcommand{\ptg}{p_{\text{\relscale{0.77}T,$\gamma$}}}
\newcommand{\ptgcut}{{p_{\text{\relscale{0.77}T,$\gamma$}}^{\textrm cut}}}
\newcommand{\ptjcut}{{p_{\text{\relscale{0.77}T,$j$}}^{\textrm cut}}}

%%newdef
\newcommand{\ptjmin}{\bar{p}_{\text{\relscale{0.77}T,$j$}}}
\newcommand{\ptgmin}{\bar{p}_{\text{\relscale{0.77}T,$\gamma$}}}
\newcommand{\ptgone}{p_{\text{\relscale{0.77}T,$\gamma_1$}}}
\newcommand{\ptgtwo}{p_{\text{\relscale{0.77}T,$\gamma_2$}}}
\newcommand{\ptgthree}{p_{\text{\relscale{0.77}T,$\gamma_3$}}}

\newcommand{\ptr}{{p_{\text{\relscale{0.77}T}}}^{\text{\relscale{0.9}r}}}
\newcommand{\ptrad}{{p_{\text{\relscale{0.77}T,rad}}}}
\newcommand{\pth}{{p_{\text{\relscale{0.77}T,$H$}}}}
\newcommand{\ptz}{{p_{\text{\relscale{0.77}T,$Z$}}}}
\newcommand{\ptw}{{p_{\text{\relscale{0.77}T,$W$}}}}
\newcommand{\ptnu}{{p_{\text{\relscale{0.77}T,$\nu$}}}}
\newcommand{\mtwz}{{m_{\text{\relscale{0.77}T,$WZ$}}}}
\newcommand{\dphiwz}{{\Delta\phi_{\text{\relscale{0.77}$WZ$}}}}
\newcommand{\dyZlW}{{|y_{\text{\relscale{0.77}$Z$}}-y_{\text{\relscale{0.77}$
\ell_W$}}|}}
\newcommand{\ptww}{{p_{\text{\relscale{0.77}T,$WW$}}}}
\newcommand{\ptwp}{{p_{\text{\relscale{0.77}T,$W^+$}}}}
\newcommand{\ptwm}{{p_{\text{\relscale{0.77}T,$W^-$}}}}
\newcommand{\mtww}{{m_{\text{\relscale{0.77}T,$WW$}}}}
\newcommand{\mtwwexp}{{m_{\text{\relscale{0.77}T,$WW$}}^{\textrm exp}}}
\newcommand{\ptllg}{{p_{\text{\relscale{0.77}T,}\ell\ell\gamma}}}
\newcommand{\ptzg}{\ptllg}
\newcommand{\ptll}{{p_{\text{\relscale{0.77}T,$e^+e^-$}}}}
\newcommand{\ptlnu}{{p_{\text{\relscale{0.77}T,$\mu\nu_\mu$}}}}
\newcommand{\ptj}{p_{\text{\relscale{0.77}T,$j$}}}
\newcommand{\ptjone}{{p_{\text{\relscale{0.77}T,$j_1$}}}}
\newcommand{\ptjoneveto}{{p_{\text{\relscale{0.77}T,$j_1$}}^{\textrm veto}}}
\newcommand{\ptjtwo}{{p_{\text{\relscale{0.77}T,$j_2$}}}}
\newcommand{\ptlm}{{p_{\text{\relscale{0.77}T,$\ell^-$}}}}
\newcommand{\ptl}{{p_{\text{\relscale{0.77}T,$\ell$}}}}
\newcommand{\ptlone}{{p_{\text{\relscale{0.77}T,$\ell_1$}}}}
\newcommand{\ptltwo}{{p_{\text{\relscale{0.77}T,$\ell_2$}}}}
\newcommand{\ptmiss}{{p_{\text{\relscale{0.77}T,miss}}}}
\newcommand{\ptmissrel}{{p_{\text{\relscale{0.77}T,miss,rel}}}}
\newcommand{\yh}{{y_{\text{\relscale{0.77}H}}}}
\newcommand{\yz}{{y_{\text{\relscale{0.77}Z}}}}
\newcommand{\ywp}{{y_{\text{\relscale{0.77}$W^+$}}}}
\newcommand{\yl}{{y_{\text{\relscale{0.77}$\ell$}}}}
\newcommand{\ylone}{{y_{\text{\relscale{0.77}$\ell_1$}}}}
\newcommand{\dphill}{{\Delta\phi_{\text{\relscale{0.77}$\ell_1\ell_2$}}}}
\newcommand{\yww}{{y_{\text{\relscale{0.77}$WW$}}}}
\newcommand{\yjone}{{y_{\text{\relscale{0.77}$j_1$}}}}
\newcommand{\dyww}{{\Delta y_{\text{\relscale{0.77}$W^-,W^+$}}}}
\newcommand{\mh}{{m_{\text{\relscale{0.77}H}}}}
\newcommand{\mz}{{m_{\text{\relscale{0.77}Z}}}}
\newcommand{\mw}{{m_{\text{\relscale{0.77}W}}}}
\newcommand{\mww}{{m_{\text{\relscale{0.77}WW}}}}
\newcommand{\mwsq}{{m^2_{\text{\relscale{0.77}W}}}}
\newcommand{\mt}{{m_{\text{\relscale{0.77}t}}}}
\newcommand{\mT}{{m_{\text{\relscale{0.77}T}}}}
\newcommand{\mgj}{{m_{\text{\relscale{0.77}$\gamma j_1$}}}}
\newcommand{\mllg}{{m_{\text{\relscale{0.77}$\ell\ell\gamma$}}}}
\newcommand{\mlnu}{{m_{\text{\relscale{0.77}$\mu\nu_\mu$}}}}
\newcommand{\etal}{{\eta_{\text{\relscale{0.77}$\ell$}}}}
\newcommand{\etallg}{{\eta_{\text{\relscale{0.77}$\ell\ell\gamma$}}}}
\newcommand{\etalone}{{\eta_{\text{\relscale{0.77}$\ell_1$}}}}
\newcommand{\etaltwo}{{\eta_{\text{\relscale{0.77}$\ell_2$}}}}
\newcommand{\etag}{{\eta_{\text{\relscale{0.77}$\gamma$}}}}
\newcommand{\etaj}{{\eta_{\text{\relscale{0.77}j}}}}
\newcommand{\etah}{{\eta_{\text{\relscale{0.77}H}}}}
\newcommand{\detallgj}{{\Delta\eta_{\text{\relscale{0.77}$\ell\ell\gamma, j_1$}}}}
\newcommand{\dphillg}{{\Delta\phi_{\text{\relscale{0.77}$\ell\ell,\gamma$}}}}
\newcommand{\drlg}{{\Delta R_{\text{\relscale{0.77}$\ell \gamma$}}}}
\newcommand{\drgjone}{{\Delta R_{\text{\relscale{0.77}$\gamma j_1$}}}}
\newcommand{\drgjtwo}{{\Delta R_{\text{\relscale{0.77}$\gamma j_2$}}}}
\newcommand{\drej}{{\Delta R_{\text{\relscale{0.77}$e j$}}}}

%new def
\newcommand{\rjg}{R_{\text{\relscale{0.77}$j \gamma$}}}
\newcommand{\rjgone}{R_{\text{\relscale{0.77}$j \gamma_1$}}}
\newcommand{\rjgtwo}{R_{\text{\relscale{0.77}$j \gamma_2$}}}
\newcommand{\rjgthree}{R_{\text{\relscale{0.77}$j \gamma_3$}}}
\newcommand{\rjgmin}{\bar{R}_{\text{\relscale{0.77}$j \gamma$}}}
\newcommand{\rjonegone}{R_{\text{\relscale{0.77}$j_1 \gamma_1$}}}
\newcommand{\rjonegtwo}{R_{\text{\relscale{0.77}$j_1 \gamma_2$}}}
\newcommand{\rjonegthree}{R_{\text{\relscale{0.77}$j_1 \gamma_3$}}}
\newcommand{\rjtwogone}{R_{\text{\relscale{0.77}$j_2 \gamma_1$}}}
\newcommand{\rjtwogtwo}{R_{\text{\relscale{0.77}$j_2 \gamma_2$}}}
\newcommand{\rjtwogthree}{R_{\text{\relscale{0.77}$j_2 \gamma_3$}}}


\newcommand{\lw}{\ensuremath{\mu}}
\newcommand{\lpw}{\ensuremath{\ell^+_{\text{\relscale{0.77}W}}}}
\newcommand{\lmw}{\ensuremath{\ell^-_{\text{\relscale{0.77}W}}}}
\newcommand{\lpmw}{\ensuremath{\ell^{\pm}_{\text{\relscale{0.77}W}}}}

\newcommand{\lz}{\ensuremath{e}}
\newcommand{\lpz}{\ensuremath{e^+}}
\newcommand{\lmz}{\ensuremath{e^-}}
\newcommand{\lpmz}{\ensuremath{e^{\pm}_{\text{\relscale{0.77}}}}}
\newcommand{\lzlead}{\ensuremath{e_{\text{\relscale{0.77}Z,1}}}}
\newcommand{\lzsubl}{\ensuremath{e_{\text{\relscale{0.77}Z,2}}}}

\newcommand{\ptlz}{\ensuremath{p_{\text{\relscale{0.77}T,\lz}}}}
\newcommand{\ptlw}{\ensuremath{p_{\text{\relscale{0.77}T,\lw}}}}
\newcommand{\ptlzlead}{\ensuremath{p_{\text{\relscale{0.77}T,\lzlead}}}}
\newcommand{\ptlzsubl}{\ensuremath{p_{\text{\relscale{0.77}T,\lzsubl}}}}

\newcommand{\mlll}{\ensuremath{m_{\text{\relscale{0.77}$3\ell$}}}}
\newcommand{\mwz}{\ensuremath{m_{\text{\relscale{0.77}WZ}}}}
\newcommand{\mtw}{\ensuremath{m_{\text{\relscale{0.77}T,W}}}}
\newcommand{\ptwz}{\ensuremath{p_{\text{\relscale{0.77}T,WZ}}}}
\newcommand{\ptlp}{\ensuremath{p_{\text{\relscale{0.77}T,\ell'}}}}
\newcommand{\dRll}{\ensuremath{\Delta R_{\text{\relscale{0.77}$\ell\ell$}}}}
\newcommand{\dRllp}{\ensuremath{\Delta R_{\text{\relscale{0.77}$\ell\ell'$}}}}
\newcommand{\etalp}{\ensuremath{\eta_{\text{\relscale{0.77}$\ell'$}}}}

\newcommand{\qcdfull}{\ensuremath{\text{NNLO}_{\textrm QCD}^{{\textrm (QCD,QED)}_{\textrm PS}}}}
\newcommand{\qcdqcd}{\ensuremath{\text{NNLO}_{\textrm QCD}^{{\textrm (QCD)}_{\textrm PS}}}}

\newcommand{\addfull}{\ensuremath{\text{NNLO}_{\textrm QCD}^{\textrm (QCD,QED)_{\textrm PS}} + \delta{\textrm NLO}_{\textrm EW}^{\textrm (QCD,QED)_{\textrm PS}}}}
\newcommand{\addqcdfull}{\ensuremath{\text{NNLO}_{\textrm QCD}^{\textrm (QCD,QED)_{\textrm PS}} + \delta{\textrm NLO}_{\textrm EW}^{\textrm (QED)_{\textrm PS}}}}
\newcommand{\addqedfull}{\ensuremath{\text{NLO}_{\textrm EW}^{\textrm (QCD,QED)_{\textrm PS}} + \delta{\textrm NNLO}_{\textrm QCD}^{\textrm (QCD)_{\textrm PS}}}}

\newcommand{\multfull}{\ensuremath{\text{NNLO}_{\textrm QCD}^{\textrm (QCD,QED)_{\textrm PS}} \times \text{K-NLO}_{\textrm EW}^{\textrm (QCD,QED)_{\textrm PS}}}}
\newcommand{\multqcdfull}{\ensuremath{\text{NNLO}_{\textrm QCD}^{\textrm (QCD,QED)_{\textrm PS}} \times \text{K-NLO}_{\textrm EW}^{\textrm (QED)_{\textrm PS}}}}
\newcommand{\multqedfull}{\ensuremath{\text{NLO}_{\textrm EW}^{\textrm (QCD,QED)_{\textrm PS}} \times \text{K-NNLO}_{\textrm QCD}^{\textrm (QCD)_{\textrm PS}}}}

\newcommand{\multmatrix}{\ensuremath{\text{NNLO}_{\textrm QCD}^{\textrm (QCD)_{\textrm PS}} \times \text{K-NLO}_{\textrm EW}^{\text{\Matrix{}}}}}
\newcommand{\QCDpEW}{\ensuremath{ \text{NNLO}_{\textrm QCD+EW}^{\textrm (QCD, QED)_{\textrm PS}}}} 
\newcommand{\QCDtEW}{\ensuremath{ \text{NNLO}_{\textrm QCDxEW}^{\textrm (QCD, QED)_{\textrm PS}}}} 
\newcommand{\QCDtEWfo}{\ensuremath{ \text{NNLO}_{\textrm QCD}^{\textrm (QCD)_{\textrm PS}} \times \text{K-NLO}_{\textrm EW}^{\textrm (f.o.)}}} 



\newcommand{\drlj}{{\Delta R_{\text{\relscale{0.77}$\ell,j$}}}}
\newcommand{\drgj}{{\Delta R_{\text{\relscale{0.77}$\gamma j$}}}}
\newcommand{\drgjo}{{\Delta R_{\text{\relscale{0.77}$\gamma j_1$}}}}
\newcommand{\drgjt}{{\Delta R_{\text{\relscale{0.77}$\gamma j_2$}}}}
\newcommand{\fcl}{{E_{\text{\relscale{0.77}T}}^{\text{\relscale{0.77}cone$0.2$}}/p_{\text{\relscale{0.77}T},\gamma}}}
\newcommand{\muF}{{\mu_{\text{\relscale{0.77}F}}}}
\newcommand{\muR}{{\mu_{\text{\relscale{0.77}R}}}}
\newcommand{\muFtwo}{{\mu^2_{\text{\relscale{0.77}F}}}}
\newcommand{\muRtwo}{{\mu^2_{\text{\relscale{0.77}R}}}}
\newcommand{\muFc}{{\mu_{\text{\relscale{0.77}F},0}}}
\newcommand{\muRc}{{\mu_{\text{\relscale{0.77}R},0}}}
\newcommand{\muRy}{{\mu_{\text{\relscale{0.77}R}}^{(0),y}}}
\newcommand{\muRb}{{\mu_{\text{\relscale{0.77}R}}^{(0),\alpha}}}
\newcommand{\KF}{K_{\text{\relscale{0.77}F}}}
\newcommand{\KR}{K_{\text{\relscale{0.77}R}}}
\newcommand{\KRy}{{K^y_{\text{\relscale{0.77}R}}}}
\newcommand{\KQ}{{K_{\text{\relscale{0.77}Q}}}}
\newcommand{\Q}{{Q_{\text{\relscale{0.77}$0$}}}}
\newcommand{\Qc}{{Q_{\text{\relscale{0.77}res},0}}}

\newcommand{\noun}[1]{{\scshape #1}}
\newcommand{\MADGRAPH}{\noun{MadGraph v4}}
\newcommand{\GOSAM}{\noun{GoSam 2.0}}
\newcommand{\POWHEG}{\noun{Powheg}}
\newcommand{\POWHEGMiNLO}{\noun{Powheg-MiNLO}}
\newcommand{\POWHEGBOX}{\noun{Powheg-Box}}
\newcommand{\POWHEGBOXRES}{\noun{Powheg-Box-Res}}
\newcommand{\POWHEGBOXVTWO}{\noun{Powheg-Box-V2}}
\newcommand{\minlobare}{{\noun{MiNLO}}}
\newcommand{\minlosimple}{{\noun{MiNLO}}}
\newcommand{\minlo}{{\noun{MiNLO$^{\prime}$}}}
\newcommand{\minnlo}{{\noun{MiNNLO$_{\textrm{PS}}$}}}
\newcommand{\GENEVA}{\noun{Geneva}}
\newcommand{\Matrix}{{\noun{Matrix}}}
\newcommand{\OpenLoops}{{\noun{OpenLoops}}}
\newcommand{\PYTHIA}[1]{\noun{Pythia{#1}}}

\newcommand{\NNLOps}{NNLO+PS}
\newcommand{\fnnlo}{NNLO}
\newcommand{\fnnnlo}{N$^3$LO}
\newcommand{\fnlo}{NLO}
\newcommand{\NLOps}{NLO+PS}
\newcommand{\setupinclusive}{{\tt inclusive setup}}
\newcommand{\setupfiducial}{{\tt fiducial setup}}
\newcommand{\setupatlas}{{\tt ATLAS setup}}

\newcommand{\abar}{\frac{\as}{2\pi}}
\newcommand{\abarmu}[1]{\frac{\as(#1)}{2\pi}}

\newcommand{\Vsc}{V}
\newcommand{\Vwa}{V_{\textrm wa}}
\newcommand{\Vfull}{V}
\newcommand{\wzc}{V_{\textrm hc}}
\newcommand{\Vr}{V_{r}}
\newcommand{\ptB}{p_{t}^{(B)}}

\newcommand{\dZ}{d{\cal Z}[\{R', k_i\}]}
\newcommand{\RpNLL}{R'_{\mathrm{NLL}}}

\newcommand{\LambdaPWG}{\Lambda_{\textrm pwg}}

\newcommand{\yll}{{y_{\text{\relscale{0.77}\ell\ell}}}}
\newcommand{\mll}{{m_{\text{\relscale{0.77}\ell\ell}}}}
\newcommand{\mQQF}{m_{Q\bar Q{\textrm F}}}
\newcommand{\muQ}{\mu_{Q}}
\newcommand{\Mdiv}{\mathcal{M}^\textrm{IR-div}}
\newcommand{\phs}{\ensuremath{\phi^{*}_\eta}\xspace}



\usepackage{xcolor}
\newcommand{\mathd}{\mathrm{d}}
\newcommand{\tmop}[1]{\ensuremath{\operatorname{#1}}}
\newenvironment{enumeratealpha}{\begin{enumerate}[a{\textup{)}}] }{\end{enumerate}}
\newenvironment{itemizedot}{\begin{itemize} \renewcommand{\labelitemi}{$\bullet$}\renewcommand{\labelitemii}{$\bullet$}\renewcommand{\labelitemiii}{$\bullet$}\renewcommand{\labelitemiv}{$\bullet$}}{\end{itemize}}
\newcommand{\Eta}{\mathrm{H}}
\newcommand{\tmverbatim}[1]{{\ttfamily{#1}}}

\def\collr{orange}
\def\colsp{blue}
\def\colmw{purple}
\def\colsk{cyan}
\def\coldr{red}
\def\colpt{red}
\def\colgz{red}
\def\coljl{magenta}
\def\colsz{violet}
\def\colcb{brown}
\def\colas{magenta}
\def\coljm{cyan}


\newcommand{\mwcom}[1]{\textit{\textcolor{\colmw}{\{MW: #1 \}}}}
\newcommand{\gzcom}[1]{\textit{\textcolor{\colpt}{\{GZ: #1 \}}}}
\newcommand{\cbcom}[1]{\textit{\textcolor{\colcb}{\{CB: #1 \}}}}
\newcommand{\ascom}[1]{\textit{\textcolor{\colas}{\{AS: #1 \}}}}
\newcommand{\jmcom}[1]{\textit{\textcolor{\coljm}{\{JM: #1 \}}}}

\newcommand{\asdel}[1]{\textcolor{\colas}{\sout{#1}}}
\newcommand{\mwdel}[1]{\textcolor{\colmw}{\sout{#1}}}
\newcommand{\gzdel}[1]{\textcolor{\colmg}{\sout{#1}}}
\newcommand{\jmdel}[1]{\textcolor{\coljm}{\sout{#1}}}

\newcommand{\mwadd}[2]{\textcolor{\colmw}{\sout{#1}#2}}
\newcommand{\asadd}[2]{\textcolor{\colas}{\sout{#1}#2}}
\newcommand{\cbadd}[2]{\textcolor{\colcb}{\sout{#1}#2}}
\newcommand{\gzadd}[2]{\textcolor{\colgz}{\sout{#1}#2}}
\newcommand{\jmadd}[2]{\textcolor{\coljm}{\sout{#1}#2}}


\def\ltap{\raisebox{-.6ex}{\rlap{$\,\sim\,$}} \raisebox{.4ex}{$\,<\,$}} 
\def\gtap{\raisebox{-.6ex}{\rlap{$\,\sim\,$}} \raisebox{.4ex}{$\,>\,$}} 
\def\lra{\leftrightarrow} 
\def\naive{na\"{\i}ve} 
%\newcommand\as{\alpha_{\mathrm{S}}} 
%\newcommand\f[2]{\frac{#1}{#2}} 
\def\bom#1{{\mbox{\boldmath $#1$}}} 
\def\to{\rightarrow}
\def\ito{\leftarrow} 
\def\nn{\nonumber} 
\def\mbbggs{m_{2b2\gamma}^{\star}}
\def\arrowlimit#1{\mathrel{\mathop{\longrightarrow}\limits_{#1}}} 
\def\ptmin{p_{T}^{\textrm min}}
\def\ptmax{p_{T}^{\textrm max}}
\def\ptveto{p_{T,\ell\ell\gamma}^{\textrm veto}}
\def\ep{\epsilon}
\def\ms{${\overline {\textrm MS}}$}
\def\perc{\%}
\def\mH{m_H}
\def\mb{m_b}
\def\mt{m_t}
\def\mw{m_W}
\def\mz{m_Z}
\def\qT{q_T}
%\def\pT{p_T}
\def\GeV{\mathrm{GeV}}
\def\TeV{\mathrm{TeV}}
\def\tL{{\widetilde L}}

\newcommand{\eqn}[1]{eq.\,(\ref{#1})}
\newcommand{\neqn}[1]{eqs.\,(\ref{#1})}
\newcommand{\fig}[1]{figure\,\ref{#1}}
\newcommand{\figs}[1]{figures\,\ref{#1}}
\newcommand{\tab}[1]{table\,\ref{#1}}
\newcommand{\sct}[1]{section\,\ref{#1}}
\newcommand{\scts}[1]{sections\,\ref{#1}}
\newcommand{\app}[1]{appendix\,\ref{#1}}

\def\refeq#1{\mbox{eq.\,\eqref{#1}}}
\def\refeqs#1{\mbox{eqs.\,\eqref{#1}}}
\def\reffi#1{\mbox{figure\,\ref{#1}}}
\def\reffitwo#1#2{\mbox{figures\,\ref{#1} and \ref{#2}}}
\def\reffis#1#2{\mbox{figures\,\ref{#1}--\ref{#2}}}
\def\refta#1{\mbox{table\,\ref{#1}}}
\def\reftatwo#1#2{\mbox{tables\,\ref{#1} and \ref{#2}}}
\def\reftas#1{\mbox{tables\,\ref{#1}}}
\def\refse#1{\mbox{section\,\ref{#1}}}
\def\refsetwo#1#2{\mbox{sections\,\ref{#1} and \ref{#2}}}
\def\refses#1{\mbox{sections\,\ref{#1}}}
\def\refapp#1{\mbox{app.\,\ref{#1}}}
\def\citere#1{\mbox{ref.\,\cite{#1}}}
\def\citeres#1{\mbox{refs.\,\cite{#1}}}


\newcommand{\rcut}{\ensuremath{r_{\mathrm{cut}}}}
%\newcommand{\zz}{\ensuremath{ZZ}}
\newcommand{\ww}{\ensuremath{W^+W^-}}
\newcommand{\wz}{\ensuremath{W^\pm Z}}
\newcommand{\wpz}{\ensuremath{W^+Z}}
\newcommand{\wmz}{\ensuremath{W^-Z}}
\newcommand{\z}{\ensuremath{Z}}
\newcommand{\w}{\ensuremath{W}}

\newcommand{\abbrev}{}
\newcommand{\llog}{\text{\abbrev LL}}
\newcommand{\nll}{\text{\abbrev NLL}}
\newcommand{\nnll}{\text{\abbrev NNLL}}
\newcommand{\lo}{\text{\abbrev LO}}
\newcommand{\nlo}{\text{\abbrev NLO}}
\newcommand{\nnlo}{\text{\abbrev NNLO}}
\newcommand{\nlonll}{\nlo\plus\nll}
\newcommand{\nnlonnll}{\nnlo\plus\nnll}
\newcommand{\qcd}{{\abbrev QCD}}
\newcommand{\D}{\mathrm{d}}

\newcommand{\cme}{centre-of-mass energy}
\newcommand{\cmes}{centre-of-mass energies}

\newcommand\Tstrut{\rule{0pt}{3.0ex}}         % = `top' strut
\newcommand\Bstrut{\rule[-1.5ex]{0pt}{0pt}}   % = `bottom' strut

\interfootnotelinepenalty=10000
%\setlength{\parindent}{0pt}

\newcommand\mlbl[1]{{\mbox{\footnotesize #1}}} 

\newcommand{\elle}{\ensuremath{\ell}}
\newcommand{\genllln}{\ensuremath{\elle\elle\elle\nu}}
\newcommand{\llln}{\elle'^\pm{\nu}_{\elle^\prime} \elle^-\elle^+}
\newcommand{\mllln}{\ensuremath{m_{\llln}}}
\newcommand{\ptllln}{\ensuremath{p_{T,\llln}}}


\setlength{\tabcolsep}{5pt}

\usepackage{etoolbox}
\makeatletter
% \tracingpatches
\patchcmd{\@sect}{#8}{\boldmath #8}{}{}
\let\ori@chapter\@chapter
\def\@chapter[#1]#2{\ori@chapter[\boldmath#1]{\boldmath#2}}
\makeatother




\newcommand{\gev}[1]{$\unit{#1}{\giga\electronvolt}$}
\newcommand{\tev}[1]{$\unit{#1}{\tera\electronvolt}$}

\newcommand{\gevm}[1]{\unit{#1}{\giga\electronvolt}}
\newcommand{\tevm}[1]{\unit{#1}{\tera\electronvolt}}

\newcommand{\half}{$\frac{1}{2}$}

\newcommand{\ddk}[1]{\frac{d^d k_{#1}}{(4\pi)^d}}
\newcommand{\sidenote}[1]{\todo[noline]{#1}}

\newcommand\calo[1]{{\cal O}\hspace{-0.2em}\left(#1\right)}

\newcommand{\cala}{{\cal A}}
\newcommand{\bbH}{\ensuremath{b\bar{b}\text{H}}}
\newcommand{\bbtoH}{\ensuremath{b\bar{b}\rightarrow\text{H}}}
\newcommand{\ttH}{\ensuremath{t\bar{t}\text{H}}}
\newcommand{\bbphi}{\ensuremath{b\bar{b}\phi}}
\newcommand{\yt}{\ensuremath{y_t}}
\newcommand{\ytsq}{\ensuremath{y_t^2}}
\newcommand{\yb}{\ensuremath{y_b}}
\newcommand{\ybsq}{\ensuremath{y_b^2}}
\newcommand{\ybyt}{\ensuremath{y_b\, y_t}}


%%% feb15, 2025 - definitions %%%%
\newcommand{\phiF}{\Phi_{\text{F}}}
\newcommand{\phirad}{\Phi_{\text{rad}}}
\newcommand{\phiFJ}{\Phi_{\text{FJ}}}
\newcommand{\obs}{\mathcal{O}}
\newcommand{\barphiF}{\bar{\Phi}_{\text{F}}}
\newcommand{\barphiFJ}{\bar{\Phi}_{\text{FJ}}}
\newcommand{\dpwg}{\Delta_{\text{pwg}}}
\newcommand{\lpwg}{\Lambda_{\text{pwg}}}
\newcommand{\ptF}{{p_{\text{\relscale{0.77}T,$F$}}}}
\newcommand{\phiFJJ}{\Phi_{\text{FJJ}}}
\newcommand{\meF}{M_{\scriptscriptstyle\mathrm F}}
\newcommand{\meFJ}{M_{\scriptscriptstyle\mathrm FJ}}

\newcommand{\phiQQF}{\Phi_{\text{X}}}
\newcommand{\phiQQFJ}{\Phi_{\text{XJ}}}
\newcommand{\phiQQFJJ}{\Phi_{\text{XJJ}}}
\newcommand{\cflavF}{c_{\text{X}}}
\newcommand{\cflavFJ}{c_{\text{XJ}}}
\newcommand{\cflavFJJ}{c_{\text{XJJ}}}

\newcommand{\msb}{\overline{\text{MS}}}

\newcommand{\muIR}{{\mu_{\text{\relscale{0.77}S}}}}
\newcommand{\muRuno}{{\mu_{\text{\relscale{0.77}R,1}}}}
\newcommand{\muRdue}{{\mu_{\text{\relscale{0.77}R,2}}}}
\newcommand{\muIRuno}{{\mu_{\text{\relscale{0.77}S,1}}}}
\newcommand{\muIRdue}{{\mu_{\text{\relscale{0.77}S,2}}}}
\newcommand{\muRunotwo}{{\mu^2_{\text{\relscale{0.77}R,1}}}}
\newcommand{\muRduetwo}{{\mu^2_{\text{\relscale{0.77}R,2}}}}
\newcommand{\muIRunotwo}{{\mu^2_{\text{\relscale{0.77}S,1}}}}
\newcommand{\muIRduetwo}{{\mu^2_{\text{\relscale{0.77}S,2}}}}


\preprint{
%  \small
\vspace{-24pt}
  \begin{flushright}
  LHCHWG-2025-\#\\
  MPP-2025-116\\
  DESY-25-093\\
  TIF-UNIMI-2025-15
  \end{flushright}
}

\title{Modelling {\boldmath{$b\bar b H$}} production for the LHC at 13.6 TeV}

\author[a]{Christian Biello,}
\author[b]{Alessandro Gavardi,}
\author[b]{Rebecca von Kuk,}
\author[c,d]{Matthew A.~Lim,}
\author[e]{Stefano Manzoni,}
\author[e]{Elena Mazzeo,}
\author[f]{Javier Mazzitelli,}
\author[a,g]{Aparna Sankar,}
\author[f]{Michael Spira,}
\author[b]{Frank Tackmann,}
\author[a]{Marius Wiesemann,}
\author[a,g]{Giulia Zanderighi,}
\author[h]{Marco Zaro}

\affiliation[a]{Max-Planck-Institut f\"ur Physik, Boltzmannstrasse 8, 85748 Garching, Germany}
\affiliation[b]{Deutsches Elektronen-Synchrotron DESY, Notkestr. 85, 22607 Hamburg, Germany}
\affiliation[c]{Department of Physics and Astronomy, University of Sussex, Sussex House, Brighton, BN1 9RH, UK}
\affiliation[d]{Università degli Studi di Milano-Bicocca \& INFN Sezione di Milano-Bicocca, Piazza della Scienza 3, Milano 20126, Italy}
\affiliation[e]{CERN, CH-1211 Geneva 23, Switzerland}
\affiliation[f]{PSI Center for Neutron and Muon Sciences, 5232 Villigen PSI, Switzerland}
\affiliation[g]{Physik Department T31, James-Franck-Straße 1, Technische Universität München, D-85748\\Garching, Germany}
\affiliation[h]{Università degli Studi di Milano \& INFN Sezione di Milano, Via Celoria 16, 20133 Milano, Italy}

\emailAdd{biello@mpp.mpg.de, marius.wiesemann@mpp.mpg.de}


\abstract{
We present new state-of-the-art predictions for Standard Model Higgs production in association with a bottom-quark pair (\bbH{}). Updated cross-sections are computed in accordance with the recommendations of the LHC Higgs Working Group with a center-of-mass energy of 13.6 TeV. For the total inclusive cross section, we provide matched predictions of the massless five-flavour scheme (5FS) and the massive four-flavour scheme (4FS) at the fixed-order level. We further present recently obtained simulations matched with parton showers in both flavour schemes within the SM, and also discuss them in the context of
potential BSM scenarios. In the massless scheme, we compare different NNLO+PS predictions obtained through the \minnlo{} and \GENEVA{} generators. 
In addition, the role of 4FS predictions is studied as a background to HH searches, considering both the top-quark and bottom-quark Yukawa contributions to $b\bar bH$ production. Finally, we analyse the sensitivity of the Higgs transverse momentum spectrum to light-quark Yukawa couplings
in the diphoton decay channel based on \minnlo{} simulations.
}


\begin{document}
\maketitle
\flushbottom

\begin{comment}
{\color{red}
\textbf{PARAMETERS AND SETTINGS from LHCH WG1+bbH reccomendations. Please consult the page: \href{https://twiki.cern.ch/twiki/bin/view/LHCPhysics/LHCHWG136TeVxsec}{WG1 Twiki}}
\begin{itemize}
	\item The bottom-quark mass is MS-bar bottom-quark mass $m_b(m_b)=(4.18\pm0.03)$\,GeV.
	\item Corresponding on-shell mass is $m_b^{OS}=(4.92\pm0.13)$\,GeV (use in 4FS).
	\item Yukawa: always use MS-bar bottom-quark mass and complex EW scheme for Higgs VEV. For the Higgs VEV, please use $m_W = 80.379$\,GeV, $\Gamma_W=2.085$\,GeV, $m_Z = 91.1876$\,GeV, $\Gamma_Z=2.4952$\,GeV, and $\alpha_{ew}=1/132.34890452162441$ in the $\alpha$ scheme. The derived coupling in complex scheme are $G_\mu=1.164363042525945\times 10^{-5}$ and vev $v=246.403285031993-i3.80597691412200$.
	\item PDFs: PDF4LHC21\_40. The 4 flavour variant should be used when relevant. The strong coupling is set to $\alpha_s(M_Z)=0.1180$.
	\item For fully-inclusive cross-sections numbers, we follow the WG1 Twiki for accessing the PDFs and alphas uncertainties. Freedom in choosing the central scales.
	\item For fully-exclusive cross-sections predictions, we use $m_H$ as the central scale of both Yukawa, $\mu_F$ and $\mu_R$ scale vatiations. Scale variations are obtained by changing the renormalisation and factorisation scales by a factor of $2$ with $\frac{1}{2}\leq\mu_R/\mu_F\leq 2$.
\end{itemize}
}
\end{comment}

\section{Introduction}

Higgs production in association with bottom quarks ($b\bar bH$) proceeds via two dominant production mechanisms within the SM. The typically 
considered $b\bar bH$ process proceeds via tree-level diagrams at LO where the Higgs couples to external bottom quarks, see \fig{fig:bbhlo} for example,
which will be referred to as {\it Higgs radiation off bottom quarks}.
Hence, the cross section is proportional to the squared bottom Yukawa coupling ($y_b^2$). The other (in the SM even larger) production 
mechanism is through the {\it loop-induced gluon fusion process}, where the Higgs couples to the quark loop and a radiated gluon splits into 
a bottom-quark pair, see \fig{fig:bbhyt}. The dominant contribution to this process is given by the top-quark loop and the cross section is therefore proportional 
to the squared top Yukawa coupling ($y_t^2$). Interference effects between the 
two production mechanisms are of order $y_t\,y_b$, but they are bottom-mass suppressed and appear only if the bottom quark is considered in a massive scheme.

We can thus write the \bbH{} cross section as follows:
\begin{equation}
\begin{split}
    {\rm d} \sigma & = \ybsq{}\,\alpha_s^2\left( \Delta_{\ybsq}^{(0)} + \mathcal O(\alpha_s) \right) +y_t y_b\, \alpha_s^3\left( \Delta_{\ybyt}^{(0)} + \mathcal O(\alpha_s)  \right) + \ytsq{} \,\alpha_s^4\left( \Delta_{\ytsq}^{(0)} + \mathcal O(\alpha_s)  \right)\, ,
\end{split}
\label{eq:hbbxsec2}
\end{equation}
where $\Delta_{X}^{(0)}$ is the LO contribution to each coupling structure. The first term corresponds to Higgs radiation off bottom quarks, the last term
to the loop-induced gluon fusion process, and the central one is their interference. Higher-order corrections, indicated generically
by the $\mathcal O(\alpha_s)$ (and higher) corrections, 
are typically computed separately to each process, with the caveat that at higher orders in QCD the two processes mix giving rise 
their $y_t y_b$ interference contributions. Notice that the loop-induced gluon fusion contribution is formally a NNLO (relative $\alpha_s^2$) correction
due to the loop suppression, which however is negated by its $y_t^2/y_b^2$ enhancement compared to the LO $y_b^2$ cross section.
The cross section of the $y_t^2$ contribution is roughly twice as large as the $y_b^2$ one.
There are further relevant \bbH{} production mechanisms, including Higgsstrahlung and vector-boson fusion, 
but their numerical impact sufficiently subleading, at least for the inclusive cross section.

For fully inclusive Higgs production the radiation off bottom quarks is about two orders of magnitude smaller than the dominant gluon-fusion cross section and ranks at the same size as $t\bar tH$ production. $b\bar bH$ production yields therefore a subleading, but yet relevant 
contribution to the inclusive Higgs cross section, especially in the precision age of the LHC. However, as soon as $b$ tagging is applied 
in experimental analyses, the $b\bar bH$ production rates are further reduced significantly so that the measurement of the \bbH{} process in the 
SM has not been achieved yet at the LHC. In the MSSM or the 2HDM of type II, on the other hand, the bottom Yukawa coupling is strongly enhanced for large values of $\tan\beta = v_2/v_1$, where $v_{1,2}$ denote the vacuum expectation values of the two scalar CP-even Higgs fields so that this production becomes the
dominant one, i.e.\ even larger than the gluon-fusion mechanism. In addition, \bbH{} production is a major background to di-Higgs searches, where at least 
one of the two Higgs bosons decays to bottom quarks.

The calculation of the $b\bar bH$ cross section can be performed in two different schemes. In the four-flavour scheme (4FS), the bottom quarks are massive and thus no bottom PDFs are taken into account so that the bottom quarks are entirely generated in the final state starting from quark-antiquark and gluon-gluon initial states at LO, see \fig{fig:bbhlo}\,(left, center). The calculation requires 4FS PDFs and a 4FS strong coupling $\alpha_s$ in order to avoid artificial large logarithms at higher orders. The 4FS calculation has been performed at NLO QCD some time ago \cite{dittmaier:2003ej,dawson:2003kb}, while NNLO QCD results became available only very recently \cite{Biello:2024pgo}.
\begin{figure}[hbt]
\begin{center}
    \includegraphics[height=2cm]{./diags/gg-bbH.pdf}\hspace*{2cm}
    \includegraphics[height=2cm]{./diags/qq-bbH.pdf}\hspace*{2cm}
    \includegraphics[height=2cm]{./diags/bb-H.pdf}
	\vspace{0.2cm}
  \caption{Typical LO Feynman diagrams contributing to $\bbH$ production in
the four-flavour scheme (left, centre) and the five-flavour scheme (right).}
  \label{fig:bbhlo}
\end{center}
\end{figure}

In the 4FS, the integration over the transverse momenta of the final-state bottom quarks, however, generates logarithmic contributions in the 
bottom mass that might reduce the perturbative convergence. In order to resum these logarithms, bottom densities need to be introduced by
treating the bottom quark as a massless particle. The DGLAP evolution of the PDFs leads to the resummation. This framework defines the five-flavour scheme (5FS) and starts from a $b\bar b$ initial state at LO, see \fig{fig:bbhlo}\,(right), which neglects the off-shellness and transverse momenta of the initial-state bottom quarks
as well as all power corrections in the bottom mass. The first two approximations are resolved by adding higher-order QCD corrections that, order by order, restore the full kinematics of the bottom quarks. However, finite bottom-mass effects cannot be studied in the 5FS, but they can be included
through a combination with the 4FS calculation. The NLO QCD cross section in the 5FS has been obtained some time 
ago \cite{dicus:1998hs,balazs:1998bm}. The NNLO QCD corrections \cite{harlander_2003} can range up to several 10\% (depending on the scale settings)
and they stabilise the scale dependence significantly. More recently, the 5FS calculation has even been extended to N$^3$LO QCD \cite{duhr:2019kwi}, which 
induces a small correction and further reduces the scale uncertainties.

At sufficiently high perturbative order, the two schemes need to approach each other. However, the comparison of the 4FS at NLO and the 5FS at NNLO shows a discrepancy at the level of 20--30\%, so that a combination of both schemes became mandatory for reliable prediction. In order to cope with the logarithmic terms in the 4FS, a factorisation scale smaller than the Higgs mass has been introduced (typically $M_H/4$) that reduces the differences between the 4FS and 5FS \cite{campbell2004,Maltoni:2012pa}. The combination of the 4FS and 5FS was first performed by applying the empirical Santander matching \cite{harlander:2011aa}, which is based on a logarithmic weighting between the 4FS and 5FS. This heuristic approach has been replaced by two proper matching procedures some years ago, the \fonll{} approach \cite{forte:2015hba,forte:2016sja} and the \nlonnllpart{} method \cite{Bonvini:2015pxa,Bonvini:2016fgf}. Both approaches rely on a systematic expansion of the 4FS parameters and 5FS PDFs using different methods to merge the two schemes without double-counting of common contributions. The 
combined results show a better agreement with the 5FS at NNLO QCD than with the 4FS at NLO QCD. This situation has changed recently with the computation
of the NNLO QCD corrections in the 4FS \cite{Biello:2024pgo}. At NNLO QCD, 4FS and 5FS predictions agree within 10\% or better, thus solving 
the previous discrepancies between the schemes.

Apart from fixed-order predictions, there has been substantial progress also in the matching of NNLO QCD corrections with parton showers (NNLO+PS) for \bbH{} production recently.
A NNLO+PS calculation in the 5FS at has been performed using the \GENEVA{} method~\cite{Gavardi:2025zpf}, while in the \minnlo{} framework~\cite{Biello:2024pgo,Biello:2024vdh} NNLO+PS predictions for both the 4FS and 5FS predictions have been achieved. 
These three implementations provide suitable NNLO event generators for the experimental analyses.

The $y_t^2$-induced contributions to the inclusive \bbH{} cross section (cf.\ \fig{fig:bbhyt}) are not well-defined in the 5FS, since, formally, the rate diverges 
for massless bottom quarks in fixed-order perturbation theory. Only by selecting bottom-flavoured jets ($b$-jets), with an appropriate
definition of the jet flavour, fixed-order predictions for the $y_t^2$ contribution in the 5FS are infra-red safe. So far, the size of the 
$y_t^2$ contribution to the \bbH{} final state has been estimated effectively only at LO in the 5FS, using the inclusive NNLO+PS 
generator for gluon fusion \cite{hamilton:2012rf,hamilton:2013fea,hamilton:2015nsa}. In this case, reshuffling of the bottom momenta onto the mass shell and matching to the parton shower, 
where the bottom quarks are considered as massive particles, renders the selection of bottom quarks in the final state finite, even if the 
underlying perturbative calculation is performed in the 5FS.

In the 4FS, on the other hand, the $y_t^2$ part of the inclusive \bbH{} cross section is finite by construction, since the bottom quarks are treated massive.
On top of that, there is the additional  $y_t\,y_b$ interference contribution in the 4FS, which vanishes in the 5FS.
The complete NLO QCD corrections to the top-Yukawa induced terms (both $y_t^2$ and $y_b y_t$ interference) 
have been obtained in the heavy-top limit (HTL) \cite{deutschmann:2018avk} and turn out to be substantial. This calculation 
has later been extended in \citere{manzoni:2023qaf} to include the parton-shower matching at NLO (NLO+PS) to study 
\bbH{} production as a background to di-Higgs searches. 
The $y_t\,y_b$ interference amounts to about 10\% \cite{dittmaier:2003ej,dawson:2003kb} compared to the inclusive $y_b^2$ cross section, while the $y_t^2$ contribution is roughly twice as large as the $y_b^2$ one \cite{deutschmann:2018avk}.
This reduces the sensitivity of the $b\bar bH$ cross section to the bottom Yukawa coupling substantially in the SM.
For heavy Higgs bosons in BSM scenarios with an enhanced bottom-Yukawa coupling, however, the radiation off bottom quarks is the dominant contribution,
while all top Yukawa induced become subleading.

\begin{figure}
\begin{center}
    \includegraphics[height=2.15cm]{./diags/gg-bbH1loop2tb_4F.pdf}\hspace{1.2cm}
    \includegraphics[height=2.15cm]{./diags/gg-bbH1loop5tb.pdf}\hspace{1.2cm}
    \includegraphics[height=2.15cm]{./diags/qq-bbH1loop2tb.pdf}
  \caption{Typical one-loop Feynman diagrams for $\bbH$ production in the four-flavour scheme involving a $\yt$ coupling.}
  \label{fig:bbhyt}
\end{center}
\end{figure}

Finally, while the discussion up to here focused on QCD corrections, it is worth mentioning that EW corrections for \bbH{} production have also been computed in the 
4FS. %\footnote{Since EW renormalisation enforces the relation between $y_b$ and $m_b$, it is not consistent to compute NLO EW corrections in the 5FS.}
In particular, they have been first evaluated for the sole $gg$-induced mechanism~\cite{Zhang:2017mdz}, and then computed for the full production
process~\cite{Pagani:2020rsg}, and they turn out to be rather small (at the few-percent level). 
In the latter case, complete-NLO corrections (including QCD and EW corrections)
have been computed. In this study it was also shown that other production mechanism, specifically Higgsstrahlung and vector boson fusion, 
can have a relevant contribution to the $b\bar b H$ rate when requiring $b$-jets, although their impact can be minimized through
suitable kinematical requirements, as shown in \citere{Grojean:2020ech}. Nevertheless, these are additional \textit{backgrounds} to the $y_b^2$ \textit{signal}, 
further reducing the sensitivity to the bottom Yukawa coupling in \bbH{} final states at the LHC.

In this contribution, we present updated cross-section predictions for Higgs production in association with bottom quarks in both the 4FS and the 5FS. 
To provide a consistent overview of state-of-the-art \bbH{} predictions at 13.6 TeV centre-of-mass energy, all results are obtained using 
a common computational setup detailed in~\sct{sec:setup}. Updated inclusive cross section predictions at 13.6 TeV are 
reported in~\sct{sec:matchedinclusivenumbers}, which are obtained by an interpolation of existing LHCHWG data of the 
matched NLO 4FS and NNLO 5FS predictions from the \nlonnllpart{} calculation at 13 and 14 TeV. 
In~\sct{sec:resummation}, we present the resummed Higgs transverse momentum spectrum in the 5FS 
at next-to-next-to-next-to leading logarithmic  (N$^3$LL$^{\prime}$) accuracy.
Section~\ref{sec:MCyb} discusses recent developments of Monte Carlo event generators and compares NNLO+PS results
from \minnlo{} and \GENEVA{} in the 5FS, consideres \minnlo{} predictions for heavy Higgs bosons in BSM scenarios, 
and subsequently examines NNLO+PS predictions from \minnlo{} in the 4FS.
In~\sct{sec:HH}, we analyze the role of \bbH{} production as a background in di-Higgs searches, 
focusing on both $y_b^2$ and $y_t^2$ contributions in phase-space regions relevant for $HH$ studies.
Finally, we present novel results for Higgs production via light-quark fusion at NNLO+PS in~\sct{sec:lightYukawa}, 
with particular emphasis on the sensitivity of the transverse momentum distribution on the  light-quark Yukawa couplings.
We summarize our findings and discuss possible future directions for improved \bbH{} predictions in \sct{sec:conclusions}.

\section{Computational setup}\label{sec:setup}
All simulations in this paper are obtained at a centre-of-mass energy of $\sqrt{s}=13.6$ TeV, employing the PDF4LHC21\_40 (LHAPDF ID \texttt{93100}) set for the parton distribution functions for massless-scheme calculations and the set PDF4LHC21\_40\_nf4 (LHAPDF ID \texttt{93500}) for simulations with massive bottom quarks~\cite{PDF4LHCWorkingGroup:2022cjn}. 

\subsection{Numerical Inputs}
The parametric inputs, including the strong coupling constant and electroweak parameters relevant for the Higgs vacuum expectation value, follow the latest recommendations from the LHC Higgs Working Group~\cite{Karlberg:2024zxx} and the Particle Data Group parameters~\cite{ParticleDataGroup:2024cfk}. For completeness, the most relevant inputs for \bbH{} predictions are reported in the following: The predictions in the massive schemes are obtained using
an on-shell value of the bottom-quark mass of $m_b^{\rm OS}=(4.92\pm0.13)$\,GeV for internal and external bottom quark lines.
The Yukawa coupling, on the other hand, is renormalized in the $\MSbar{}$ scheme using an input value of $m_b(m_b)=(4.18\pm0.03)$\,GeV
for the renormalisation group running of the Yukawa mass. This scheme choice for the Yukawa is crucial in order to evaluate it at its natural
scale, which is of order of the Higgs boson mass, and to avoid large logarithmic corrections. The scale settings are discussed in the following subsection.
The following EW input parameters are used: $m_W = 80.379$\,GeV, $\Gamma_W=2.085$\,GeV, $m_Z = 91.1876$\,GeV, $\Gamma_Z=2.4952$\,GeV. Working in the $G_\mu$ scheme with the input value $G_\mu=1.16436\,\text{GeV}^{-2}$\,, we obtain the electromagnetic coupling $\alpha=1/132.3489045$. Employing complex masses, the corresponding vacuum expectation value is $v=(246.403-3.8060\,i)$\,GeV.
The Higgs boson is considered to be stable in all simulations,
except where its decay into photons or tau leptons is explicitly considered. 
Also in these cases it is treated as on-shell and with zero width.
The strong coupling constant is set to $\alpha_s(m_Z) = 0.1180$ 
in the 5FS calculations. The corresponding 4FS value is obtained 
by accounting for heavy-quark decoupling effects, as discussed below.
The strong coupling is obtained directly from the corresponding PDF set
for consistency.

\subsection{Running of the bottom Yukawa}
The Yukawa coupling is renormalized in the \MSbar{} scheme in both the 4FS and the 5FS, adapting the following recommendation: We derive the coupling $y_b(\muR)=m_b(\muR)/v$ by 
evolving $\hat{m}_b\equiv m_b(m_b)=4.18$\,GeV to $m_b(\muR)$ via four-loop running, solving the Renormalisation Group Equation (RGE) \cite{harlander:2002wh,baikov_2014}. 
This evolution is directly related to the one of the strong coupling constant, which is evaluated both at $\muR$ and at $\hat{m}_b$. 
In the massless scheme, these values are straightforwardly obtained by evolving the input $\alpha_s(m_Z) = 0.1180$ with four-loop QCD running and $n_f = 5$ active flavours.
In the massive scheme, the procedure follows the approach commonly adopted in modern PDF evolution libraries: starting from $\alpha_s(m_Z) = 0.1180$ we evolve down to $\hat{m}_b$
with $n_f = 5$, where the decoupling relation~\cite{vogt:2004ns} is applied to extract $\alpha_s(\hat{m}_b)$ in the 4FS. 
This value is then used as a boundary condition to evolve to $\alpha_s(\muR)$ with $n_f = 4$ active flavours.

For consistency, scale uncertainties are obtained by varying the scale with respect to the central one using the order of the evolution consistent with the 
calculation, i.e.\ two-loop running for NLO predictions and three-loop running for NNLO calculations, and with the number of active flavours consistent with 
the flavour scheme under consideration.

Predictions involving the top-quark Yukawa contribution employ the on-shell definition of the coupling. A top-quark mass of $172.5$\,GeV is used throughout the simulations.

\subsection{Scale settings}
In the Monte Carlo simulations with massless bottom quarks, we consider the Higgs mass as the central value for the factorisation ($\muF$) and renormalisation ($\muR$) scales. The theory uncertainty is estimated using the standard 7-point variation, i.e.\ by changing the scales by a factor of 2 with the constraint $\frac{1}{2}\leq\muR/\muF\leq 2$ in order to avoid large logarithmic effects. The renormalisation scales associated with the Yukawa coupling and the strong coupling constant are simultaneously by rescaling both with the same factor. We have performed detailed studies on the impact of this correlation and found that such a correlated variation provides a reliable estimate of the overall renormalisation uncertainty.

In earlier 4FS predictions, a dynamical scale—specifically, one quarter of the transverse mass of the \bbH{} system—was often adopted. Such small scale
resulted in higher 4FS cross sections, which reduced the gap with the 5FS predictions. However, with the inclusion of NNLO QCD corrections in the 4FS, 
such a scale choice is no longer necessary. Moreover, as shown in~\citere{Biello:2024pgo}, comparisons between dynamical and fixed scales at NNLO 
show minimal differences. Consequently, we recommend aligning the scale choices in the 4FS and 5FS at NNLO QCD accuracy, with the Higgs-boson 
mass providing a natural and consistent choice.

\section{Updated inclusive cross sections at 13.6 TeV}
\label{sec:matchedinclusivenumbers}
In this section, we report predictions for the inclusive \bbH{}  cross section at 13.6 TeV. 
The best description of the \bbH{} process is obtained by combining the calculations in the massless and massive schemes. The 5FS computation includes 
the resummation of large collinear logarithmic contributions, while the 4FS captures all mass effects order by order in perturbative QCD. 
As already introduced in the intrdoduction, two main approaches exist that have been applied to obtain the \bbH{} cross section at 
high accuracy: \fonll{}~\cite{forte:2015hba,forte:2016sja,Duhr:2020kzd} and \nlonnllpart{}~\cite{Bonvini:2015pxa,Bonvini:2016fgf}. 

FONLL-B~\cite{forte:2015hba,forte:2016sja} and \nlonnllpart{}~\cite{Bonvini:2015pxa,Bonvini:2016fgf} match the NNLO 5FS and the NLO 4FS cross section.
Both approaches yield results that are fully consistent with each other, as shown in~\citere{LHCHiggsCrossSectionWorkingGroup:2016ypw}. 
The novel FONLL-C matching~\cite{Duhr:2020kzd} adds the \fnnnlo{} corrections in the 5FS to FONLL-B. The \nlonnllpart{} combination introduces a resummation scale $\muB$, which enables an estimate of the 
matching uncertainties.
The $y_by_t$  contribution is easily included as a fixed-order non-singular term, since these interference effects are power corrections that 
vanish to all orders in the small bottom-quark mass limit.

The setup of the calculation follows that described in~\sct{sec:setup}, with the only difference being the choice of the central scales, which are set to
\begin{align}
\muF=\frac{1}{4}(m_H+2m_b),\hspace{1cm}\muR=\frac{1}{2}m_H\,,
\end{align}
in order to ensure a better perturbative convergence.

In~\tab{tab:bbH136lin}, we report the results of the \nlonnllpart{} combination via a linear interpolation between the existing 13 TeV and 14 TeV results, using the relation
\begin{align}
\sigma(13.6\,\text{TeV})=0.4\sigma(13.0\,\text{TeV})+0.6\sigma(14.0\,\text{TeV})\,.
\end{align}
The cross section is provided together with an overall theoretical uncertainty. This includes the standard 7-point renormalisation and factorisation scale variation, the resummation scale uncertainty, and the uncertainties associated with the parton distribution functions (PDFs) and the strong coupling constant. An alternative cross-section estimation is presented in~\tab{tab:bbH136log} where the same numbers have been combined using a logarithmic interpolation. In order to test the accuracy of the interpolation, we have also considered linear interpolation of 13.5 TeV and 14 TeV results. Using the latter, we have estimated the cross-section in~\tab{tab:bbH136linalt} with the following linear interpolation,
\begin{align}
	\sigma(13.6\,\text{TeV})=0.8\sigma(13.5\,\text{TeV})+0.2\sigma(14.0\,\text{TeV})\,.
\end{align}
We observe a good agreement between the cross-section values obtained with different interpolations or different choices of boundary conditions. Due to the good accuracy of the interpolation and the substantial theoretical and parametric uncertainty, dedicated runs at 13.6 TeV are not required.

\begin{table}[ht!]
\begin{center}%
\begin{small}%
\tabcolsep5pt
\begin{tabular}{cccc}%
$m_h$[GeV] & $\sigma^{}$[fb] & $\Delta_{\left(\mu_{R},\mu_{F}\right)\oplus\mu_{B}}$[\%] & $\Delta_{\mathrm{PDF}\oplus\alpha_s}$[\%]  \\\hline
$125.00$ & $569.1$ & $\pm8.63$ & ${{+3.35}}/{-3.44}$ \\
$125.09$ & $567.9$ & $\pm8.63$ & ${{+3.36}}/{-3.44}$ \\
$125.10$ & $567.7$ & $\pm8.63$ & ${{+3.36}}/{-3.44}$ \\
$125.20$ & $566.3$ & $\pm8.63$ & ${{+3.35}}/{-3.44}$ 
\end{tabular}%
\end{small}%
\end{center}%
\caption{Total \bbH{} cross sections in the SM for a LHC CM energy of $\sqrt{s}=13.6$ TeV obtained via linear interpolation of results from 13 TeV and 14 TeV. The results are given with symmetrised uncertainties from the 7-point scale variation combined with the resummation dependency and the parametric uncertainty from PDFs and strong coupling.}
\label{tab:bbH136lin}
\end{table}

\begin{table}[ht!]
\begin{center}%
\begin{small}%
\tabcolsep5pt
\begin{tabular}{cccc}%
$m_h$[GeV] & $\sigma^{}$[fb] & $\Delta_{\left(\mu_{R},\mu_{F}\right)\oplus\mu_{B}}$[\%] & $\Delta_{\mathrm{PDF}\oplus\alpha_s}$[\%]  \\\hline
$125.00$ & $568.1$ & $\pm8.64$ & ${{+3.35}}/{-3.44}$ \\
$125.09$ & $566.9$ & $\pm8.64$ & ${{+3.36}}/{-3.44}$ \\
$125.10$ & $566.7$ & $\pm8.64$ & ${{+3.36}}/{-3.45}$ \\
$125.20$ & $565.3$ & $\pm8.64$ & ${{+3.35}}/{-3.45}$ 
\end{tabular}%
\end{small}%
\end{center}%
\caption{Total \bbH{} cross sections in the SM for a LHC CM energy of $\sqrt{s}=13.6$ TeV obtained via logarithmic interpolation of results from 13 TeV and 14 TeV.}
\label{tab:bbH136log}
\end{table}

\begin{table}[ht!]
\begin{center}%
\begin{small}%
\tabcolsep5pt
\begin{tabular}{cccc}%
$m_h$[GeV] & $\sigma^{}$[fb] & $\Delta_{\left(\mu_{R},\mu_{F}\right)\oplus\mu_{B}}$[\%] & $\Delta_{\mathrm{PDF}\oplus\alpha_s}$[\%]  \\\hline
$125.00$ & $567.7$ & $\pm8.86$ & ${{+3.30}}/{-3.68}$ \\
$125.09$ & $566.4$ & $\pm8.92$ & ${{+3.43}}/{-3.51}$ 
\end{tabular}%
\end{small}%
\end{center}%
\caption{Total \bbH{} cross sections in the SM for a LHC CM energy of $\sqrt{s}=13.6$ TeV obtained via linear interpolation of results from 13.5 TeV and 14 TeV.}
\label{tab:bbH136linalt}
\end{table}

\section{Transverse-momentum resummation at third order}
\label{sec:resummation}
The Higgs transverse momentum spectrum provides a promising approach for extracting the Yukawa coupling from Higgs production processes, because its shape is sensitive to the precise value of the coupling.
In particular, one can exploit the pattern of QCD emissions from the incoming quarks and gluons to discriminate between the gluon and various quark channels in the initial state~\cite{Ebert:2016idf}.
That is, the radiation pattern for different initial states yields different shapes for the transverse momentum ($q_T$) spectrum of the recoiling Higgs boson. As a result, a precise measurement and fit to the Higgs $q_T$ spectrum, especially at small $q_T$, allows one to gain sensitivity to the quark Yukawa couplings~\cite{Bishara:2016jga, Soreq:2016rae}.
At small $q_T \ll m_H$,
this requires the all-order resummation of logarithms of $q_T/m_H$ that would
otherwise spoil the convergence of perturbation theory in this regime.

In this section, we provide recent results for the resummed $q_T$ spectrum for $b\bar{b}\to H$ at N$^3$LL$^{\prime}$ order matched to fixed NNLO and approximate N$^3$LO in the 5FS \cite{Cal:2023mib}. Previously the spectrum was calculated to NNLL$^{\prime}+$NNLO accuracy\footnote{We use the logarithmic counting of ref.~\cite{Berger:2010xi}, where the prime denotes inclusion of higher order boundary terms.} ~\cite{Harlander:2014hya} using the CSS formalism~\cite{Collins:1981uk,Collins:1981va, Collins:1984kg}. For this prediction, we use soft-collinear effective theory (SCET)
\cite{Bauer:2000yr,Bauer:2001ct,Bauer:2001yt,Bauer:2002nz,Beneke:2002ph} to resum the logarithms of $q_T/m_H$ which is equivalent to a modern formulation of the CSS formalism.
We employ the rapidity
renormalisation group~\cite{Chiu:2012ir} together with the
exponential regulator~\cite{Li:2016axz} for which the ingredients
required for the resummation at N$^3$LL$^{\prime}$ are known.
The singular cross section can be written in a factorized form as
\begin{align} \label{tmd_factorization}
\frac{\mathrm{d} \sigma^\mathrm{sing}}{\mathrm{d} Y \mathrm{d}^2 \qt}
&= \sum_{a,b} H_{ab}(m_H^2; \mu) [B_a\otimes  B_b \otimes S_{ab}]( x_a, x_b, \qt; \mu)
\,,\end{align}
%%%
where the kinematic quantities $\omega_{a,b}$ and $x_{a,b}$ are given by
%%%
\begin{align}
\omega_{a}=m_H e^{+ Y}, \quad \omega_{b}=m_H e^{-Y} \quad \text{ and} \quad \quad x_{a,b}= \frac{\omega_{a,b}}{E_{cm}}
\,.\end{align}
%%%
The hard function $H_{ab}$ is process dependent and describes physics at the hard scale $\mu\sim m_H$. The beam functions $B_{a,b}$ and the soft function $S_{ab}$ describe collinear and soft radiation at the low scale $\mu \sim q_T$. 
The term $[B_a\otimes  B_b \otimes S_{ab}]$ is usually evaluated in Fourier-conjugate $b_T$ space, as the convolutions in $q_T$ in
\eqn{tmd_factorization} turn into products in $b_T$ space.

\begin{figure}[t!]
	\begin{center}
		\begin{tabular}{cc}
			\includegraphics[width=.55\textwidth, page=1]{plots/5fs/resummation/ptzoom_Higgs__resum.pdf}
		\end{tabular}
		\vspace*{1ex}
		\caption{Higgs transverse-momentum spectrum predicted by the analytic resummed prediction at $\text{NNLO+NLO}$ (orange, dashed), $\text{N}^3\text{LL+NNLO}$ (blue, dotted) and $\text{N}^3\text{LL$^{\prime}$+aN}^3\text{LO}$ (green, solid) accuracy.\label{fig:resumplot}}
	\end{center}
\end{figure}
To perform the all-order resummation, each function is first evaluated at its own
natural boundary scale(s): $\mu_H$, $(\mu_B, \nu_B)$, and $(\mu_S, \nu_S)$. By
choosing appropriate values for the boundary scales close to their canonical
values , each function is free of large logarithms and can
therefore be evaluated in fixed-order perturbation theory. Next, all functions
are evolved from their respective boundary conditions to a common arbitrary point
$(\mu, \nu)$ by solving their coupled system of renormalisation group equations. For more details we refer to Refs.~\cite{Ebert:2016gcn,Ebert:2020dfc,Cal:2023mib}.
For the resummation at N$^3$LL$^{\prime}$ we require the N$^3$LO
boundary conditions for the hard function~\cite{Gehrmann:2014vha,
	Ebert:2017uel}, and the beam and soft functions~\cite{Lubbert:2016rku,
	Li:2016ctv, Billis:2019vxg, Luo:2019szz, Ebert:2020yqt}. We also need the 3-loop
noncusp virtuality anomalous dimensions~\cite{Lubbert:2016rku, Moch:2005id, Stewart:2010qs,
	Bruser:2018rad, Billis:2019vxg} and rapidity anomalous
dimension~\cite{Lubbert:2016rku, Li:2016ctv, Vladimirov:2016dll}, as well as
the 4-loop cusp anomalous dimension
$\Gamma_\mathrm{cusp}$~\cite{Korchemsky:1987wg, moch:2004pa, Bruser:2019auj,
	Henn:2019swt, vonManteuffel:2020vjv} and QCD $\beta$
function~\cite{Tarasov:1980au, Larin:1993tp, vanRitbergen:1997va,
	Czakon:2004bu}.

To arrive at a consistent prediction, we need to match the N$^3$LL$^{\prime}$ resummed cross section to a NNLO Higgs+jet prediction for which we rely on an approximation. 
Our aN$^3$LO fixed-order cross section must contain the correct singular terms which are part of $\mathrm{d} \sigma^\mathrm{sing}$.
Further, there are large cancellations between the singular and the non-singular cross sections at large values of $q_T$ which must not be spoil in the approximation procedure.
To satisfy both requirements,
we developed a general method to decorrelate the singular and nonsingular contributions in Ref.~\cite{Cal:2023mib} which involves shifting a correlated piece between the singular and the non-singular~\cite{Dehnadi:2022prz}.
After the decorrelation procedure, we perform a Pad\'e-like approximation for the nonsingular $\mathcal{O}\left(\alpha_s^3\right)$ coefficient. 

Our numerical results for the
resummed and fixed-order singular contributions are obtained with
\textsc{SCETlib}~\cite{scetlib}.
In \fig{fig:resumplot}, we show the resummed $q_T$ spectrum for $b\bar{b}\to H$ at different resummation orders up to the highest N$^3$LL$^{\prime}+$aN$^3$LO. The bands show the perturbative uncertainty estimate. For a detailed breakdown of the uncertainties, we refer to Ref.~\cite{Cal:2023mib}. We observe excellent perturbative convergence, with reduced uncertainties at each higher order. 
%In Ref.~\cite{Cal:2023mib}, we also provide the $q_T$ spectra for $c\bar{c}\to H$ and $s\bar{s}\to H$ at N$^3$LL$^{\prime}+$aN$^3$LO which are shown in section~\ref{sec:lightYukawa}.

%%%%%%%%% MONTE CARLO SECTION %%%%%%%%%%
\section{Monte Carlo simulations at NNLO+PS}\label{sec:MCyb}
This section highlights recent advancements in Monte Carlo simulations for modelling the the $y_b^2$ contribution to \bbH{} production. The first matching with parton showers was carried out in the 5FS scheme in~\citere{Biello:2024vdh} using the \minnlo{} method. More recently, the \GENEVA{} approach has been applied to 
simulate \bbtoH{} production in the 5FS~\cite{Gavardi:2025zpf}. In this section, we present the numerical comparison of the two matching 
methods for this process using the LHCHWG setup.

We continue by discussing BSM studies using the \minnlo{} generator for an example scenario in the Minimal Supersymmetric Model (MSSM) used as a proof-of-concept. 
The last part of the section is dedicated the recently developed NNLO+PS Monte Carlo generator in the 4FS using the \minnlo{} method~\cite{Biello:2024pgo}, 
which provides the first calculation of the NNLO QCD corrections in the massive scheme. In this context, we compare Monte Carlo predictions from the 
4FS and 5FS \minnlo{} generators.

\subsection{NNLO+PS predictions in the 5FS}\label{sec:5FSNNLOPS}
\subsubsection{\minnlo{} and \GENEVA{}  methods in a nutshell}\label{sec:nutshell}

\minnlo{} \cite{Monni:2019whf,Monni:2020nks} and \GENEVA{} \cite{Alioli:2012fc,Alioli:2013hqa} are two methods that provide a consistent matching
between NNLO QCD corrections and parton showers. They both rely on the
combination of fixed-order calculations and analytic resummation
in the relevant resolution variables to achieve NNLO accuracy.
Both have have been proven to work with different types of resolution variables \cite{alioli:2021qbf,Gavardi:2023aco,Ebert:2024zdj,Gavardi:2025zpf}
and they have been successfully applied to obtain NNLO+PS predictions for many processes of colour-singlet production \cite{Lombardi:2020wju,Lombardi:2021rvg,Buonocore:2021fnj,Lombardi:2021wug,Zanoli:2021iyp,Gavardi:2022ixt,Haisch:2022nwz,Lindert:2022qdd,Niggetiedt:2024nmp,Biello:2024vdh,Alioli:2015toa,Alioli:2019qzz,Alioli:2020qrd,alioli:2021qbf,Alioli:2021egp,Alioli:2022dkj,Alioli:2023har,Gavardi:2023aco,Gavardi:2025zpf,Alioli:2025xcu}.
The \minnlo{} approach has also been extended and applied to heavy-quark pair production at NNLO+PS \cite{mazzitelli:2020jio,mazzitelli:2021mmm,Mazzitelli:2023znt}
as well as heavy-quark pair production in association with colour singlets \cite{mazzitelli:2024ura,Biello:2024pgo}.
The two methods follow different philosophies to reach NNLO accuracy, whose
fundamentals are presented below.

We start by briefly reviewing the \minnlo{} method for colour-singlet production. We refer to \citeres{Monni:2019whf,Monni:2020nks,Ebert:2024zdj} 
for further details. The \minnlo{} procedure is derived from the differential matching of the resummation 
formula in a suitable jet-resolution variable with the fixed-order prediction at NNLO.
By choosing a specific matching scheme, where the Sudakov is factored out, the 
cross section is cast into a form that has several important features. The structure
mimics the one of the parton shower, NNLO accuracy is achieved inclusive over
first and second radiation, and due to the Sudakov suppression no slicing cut-off
is required, making the procedure very efficient for event generation.
The matching to the parton shower is then performed for the second radiation
by means of the \POWHEG{} method \cite{frixione:2007vw}.

In practice, one starts from an NLO+PS \POWHEG{} event
generator~\cite{alioli:2010xd} for colour-singlet $F$ plus one jet $J$
production, which provides the differential description of the 
second radiation, consistently matched to subsequent emissions
by the shower. Such generator includes the NLO $FJ$ cross section 
(inclusive over the second radiation), which is governed by the following
function:
\begin{equation}
  \bar{B} = \frac{d\sigma_{\scriptscriptstyle\rm
      FJ}^{(1)}}{d\Phi_{\scriptscriptstyle\rm FJ}}\!\left(\mu\right) +
  \frac{d\sigma_{\scriptscriptstyle\rm
      FJ}^{(2)}}{d\Phi_{\scriptscriptstyle\rm FJ}}\!\left(\mu\right)\,.
\end{equation}
This formula can now be extended by means of the \minnlo{} procedure,
which replaces the NLO $FJ$ cross section by the NNLO $F$ cross section
with the overall Sudakov form factor in the jet-resolution variable, mimicking
the first shower emission.
\begin{equation}
  \label{MiNNLOPS_cross_section}
  \bar{B} = e^{-\tilde{S}\left(p_{\scriptscriptstyle\rm T}\right)}
  \left\{\frac{d\sigma_{\scriptscriptstyle\rm
      FJ}^{(1)}}{d\Phi_{\scriptscriptstyle\rm
      FJ}}\!\left(p_{\scriptscriptstyle\rm T}\right) \left[1 +
    \tilde{S}^{(1)}\!\left(p_{\scriptscriptstyle\rm T}\right)\right] +
  \frac{d\sigma_{\scriptscriptstyle\rm
      FJ}^{(2)}}{d\Phi_{\scriptscriptstyle\rm
      FJ}}\!\left(p_{\scriptscriptstyle\rm T}\right) + D^{(\geq
    3)}\!\left(p_{\scriptscriptstyle\rm T}\right)
  F^{\scriptscriptstyle\rm corr}\!\left(\Phi_{\scriptscriptstyle\rm
    FJ}\right)\right\},
\end{equation}
Note that it is crucial to choose a jet-resolution variable 
consistent with the ordering variable in the parton shower in order not to break
its logarithmic accuracy. For instance, in the case of standard leading logarithmic (LL) 
transverse-momentum ordered showers (which is the current default, and 
considered throughout here) the transverse 
momentum of the colour singlet fulfills this criterion.
Here, $D^{(\geq 3)}$ contains the relevant terms to reach 
NNLO accuracy for observables inclusive over first and second radiation.
Without these corrections the inclusive observables would only be NLO accurate,
as originally introduced in the \minlo{} approach \cite{}.

The original \minnlo{} formulation was based on transverse-momentum resummation \cite{Monni:2019whf,Monni:2020nks}, but it 
can be applied, in principle, to any resolution variable. For instance, it was explicitly derived for N-jettiness in \citere{Ebert:2024zdj},
which however formally breaks the LL accuracy of the shower, due to mismatch of the resummation and the shower ordering variable. 
Beyond colour-singlet production, the \minnlo{} method has been extended to heavy-quark pair production without \cite{mazzitelli:2020jio,mazzitelli:2021mmm}
and with extra colour singlets in the final state \cite{mazzitelli:2024ura}. This advancement enables NNLO+PS predictions for \bbH{} production
not only in the massless scheme, but also in the massive one, allowing for a complete description of bottom-quark kinematics, as discussed in~\sct{sec:bbH4FS}.

The \GENEVA{} approach also builds on the analytic resummation in 
jet-resolution variables. The matching in both first 
and second radiation in \GENEVA{} are achieved through the 
analytic resummation, opposed to \minnlo{} where the second 
radiation is matched through \POWHEG{}, as discussed before.
\GENEVA{} generates the partonic events with the aim of not distorting
the spectrum in the a zero-jet resolution variable (here generically called $r_0$).
Conceptually, the \GENEVA{} method boils down to three steps.
\begin{enumerate}
\item Matching. The $r_0$ spectrum at fixed order, differential over
  the phase space with no final state partons $\Phi_0$, is matched
  additively to its analytic resummation, obtaining
  \begin{equation}
    \frac{d\sigma}{d\Phi_0 \> dr_0} =
    \frac{d\sigma^{\scriptscriptstyle\rm FO}}{d\Phi_0 \> dr_0} +
    \frac{d\sigma^{\scriptscriptstyle\rm res}}{d\Phi_0 \> dr_0} -
    \left. \frac{d\sigma^{\scriptscriptstyle\rm res}}{d\Phi_0 \>
      dr_0}\right|_{\scriptscriptstyle\rm FO}.
  \end{equation}
  The main difference to the \minnlo{} matching is that the 
  matching of the resummation is kept in this additive form, instead 
  of factoring out the overall Sudakov form factor $e^{-\tilde{S}}$ in \eqn{MiNNLOPS_cross_section}.
\item Slicing. A slicing scale $r_0^{\scriptscriptstyle\rm cut}$ is
  introduced to separate events where the radiation is not resolved
  from those where it is, whose distributions are described
  respectively by
  \begin{equation}
    \frac{d\sigma_0}{d\Phi_0}\!\left(r_0^{\scriptscriptstyle\rm
      cut}\right) = \int_0^{r_0^{\scriptscriptstyle\rm cut}} dr_0 \>
    \frac{d\sigma}{d\Phi_0 \> dr_0}
    %
    \quad \mbox{and} \quad
    %
    \frac{d\sigma_{\geq 1}}{d\Phi_0 \> dr_0} = \frac{d\sigma}{d\Phi_0
      \> dr_0} \> \theta\!\left(r_0 - r_0^{\scriptscriptstyle\rm
      cut}\right).
  \end{equation}
  This step is not required in the \minnlo{} approach, where the
  exponential suppression  by the overall Sudakov form
  factor makes the differential cross section numerically integrable
  down to $p_{\scriptscriptstyle\rm T} = 0$ (after regulating the Landau
  pole).
\item Splitting. The ${\cal P}_{0 \to 1}$ function is introduced to
  make the $r_0$ spectrum differential over the $\Phi_1$ phase space
  with one final state parton, thus obtaining
  \begin{equation}
    \frac{d\sigma_{\geq 1}}{d\Phi_1} =
    \left\{\frac{d\sigma^{\scriptscriptstyle\rm FO}}{d\Phi_1} +
    \left[\frac{d\sigma^{\scriptscriptstyle\rm res}}{d\Phi_0 \> dr_0}
      - \left. \frac{d\sigma^{\scriptscriptstyle\rm res}}{d\Phi_0 \>
        dr_0}\right|_{\scriptscriptstyle\rm FO} \right] {\cal P}_{0
      \to 1}\!\left(\Phi_1\right)\right\} \theta\!\left(r_0 -
    r_0^{\scriptscriptstyle\rm cut}\right).
  \end{equation}
  Here, ${\cal P}_{0 \to 1}$ plays the corresponding role as
  $F^{\scriptscriptstyle\rm corr}$ in
  \eqn{MiNNLOPS_cross_section}.
\end{enumerate}
The second emission is then generated by iterating this procedure 
for an one-jet resolution variable $r_1$ on
the $d\sigma_{\geq 1} / d\Phi_1$ differential cross section at NLO. Most
\GENEVA{} implementations use the 0- and 1-jettiness $r_0=\mathcal{T}_0$
and $r_1=\mathcal{T}_1$ as resolution variables. The first implementation
that adopted the colour-singlet transverse momentum
$r_0=q_{\scriptscriptstyle\rm T}$ instead was presented
in~\citere{alioli:2021qbf}. This approach was refined and extended
in~\citere{Gavardi:2025zpf} (also using a $p_{\scriptscriptstyle\rm
  T}$-like observable as 1-jet resolution variable $r_1$), where the \bbtoH{} process was discussed.

A subtlety related to the slicing of the phase space in \GENEVA{} is
that, since the scales appearing in the resummed cumulant and spectrum
are a function of the resolution variable $r_0$, the operations of
setting the scales and integrating the spectrum do not commute. This
leaves us with two main options. Setting the scales in the spectrum
provides the best theoretical description of the $r_0$ distribution in
the region of small $r_0$ at the price of introducing spurious
subleading contributions in the distributions inclusive over the
radiation. Setting the scales in the cumulant and defining the
spectrum as its derivative, on the other hand, enforces the generated
events to reproduce the exact NNLO inclusive distributions by
construction. The numerical difference between the two choices is discussed in the following section, specifically in \fig{fig:genevaptH}.

Besides treating the resolution variables in different ways, the two
approaches may also produce small differences in the distributions
inclusive over the radiation. The \minnlo{} approach can differ from 
fixed-order NNLO predictions by terms beyond the nominal accuracy,
due to the specific matching scheme, where the Sudakov is factored out.
 The additive approach
adopted by \GENEVA{}, on the other hand, can exactly align 
terms beyond accuracy with the fixed-order differential NNLO cross section.
On the other hand, the presence of the overall Sudakov form factor in
\minnlo{} has the advantage of a better numerical efficiency with a small number 
negative weights and without having to deal with large cancellations in a
slicing cutoff, as introduced in \GENEVA{}. The distribution in the QCD emissions 
(which also have lower perturbative accuracy) is instead more heavily
dependent on the specifics of the two approaches, and differences between
the methods can be used estimate the respective theoretical
uncertainties. Despite these differences, \minnlo{} and \GENEVA{} are generally 
agree within their respective uncertainties, especially for observables inclusive 
over radiation, where any difference is beyond NNLO QCD and should be 
covered by scale uncertainties.

\subsubsection{Numerical comparison of \minnlo{} and \GENEVA{} predictions}
In this section, we present a numerical comparison between two Monte Carlo (MC) generators for \bbtoH{} production in the 5FS in the \minnlo{} and \GENEVA{} frameworks. The settings adopted are described in~\sct{sec:setup}, with the only notable difference being the treatment of the running bottom-quark Yukawa coupling. 
The \minnlo{} predictions follow the setup in \sct{sec:setup} using a four-loop running 
\cite{harlander:2002wh,baikov_2014} to evolve the bottom Yukawa from 
the input to the hard scale using, while the \GENEVA{} generator employs a three-loop running throughout derived from SCET. This difference is numerically very small and can be safely neglected in the following comparison. Both the MC generators produce events that are showered using~\PYTHIA{8} with a local recoil.

To account for resummation scale uncertainties, the \minnlo{} results are obtained using the default central value of $\KQ = 0.25$, with the alternative choice $\KQ = 0.5$ included in the uncertainty estimate. The overall theoretical uncertainty is assessed via a 14-point scale variation, combining standard renormalisation and factorisation scale variations with the additional resummation scale variation. In the matching region, the logarithmic contributions are smoothly suppressed towards large transverse momenta using modified logarithms with $p=6$, as originally introduced in~\citere{Monni:2019whf}. We have tested alternative parameters for the modified logarithms and found only minor effects, which are well within the \minnlo{} scale uncertainties.

The \GENEVA{} generator uses the cumulant scale choices in order to have a better agreement of inclusive observables with NNLO QCD predictions. Moreover, it applies default theoretical-uncertainty estimation with the included matching uncertainty as discussed in~\citere{Gavardi:2025zpf}.



\begin{figure}[t!]
\begin{center}
\begin{tabular}{cc}
\includegraphics[width=.45\textwidth, page=1]{plots/5fs/genevaminnlo/minnloKQvar-geneva-ptH.pdf}&
\includegraphics[width=.45\textwidth, page=1]{plots/5fs/genevaminnlo/minnloKQvar-genevaspec-ptHzoom.pdf}
\end{tabular}
\vspace*{1ex}
\caption{Comparison of \minnlo{} (blue, dashed) and \GENEVA{} (brown, dotted) predictions at NNLO+PS level for the transverse momentum distribution of the Higgs boson, where \GENEVA{} uses their default cumulant scale choice. The zoomed version on the right shows also the alternative spectrum scale choice of the \GENEVA{} generator (red, solid).\label{fig:genevaptH}}
\end{center}
\end{figure}

We start the comparison by focusing on the transverse momentum spectrum of the Higgs boson in~\fig{fig:genevaptH}. The left plot shows the spectrum in a large range. As 
expected, we observe good agreement between \minnlo{} and \GENEVA{} predictions
at large transverse momenta, where both calculations are effectively NLO accurate and 
governed by $H$+jet production. Also at the peak both generators yield relatively 
close results, consistent within their respective uncertainty. However, directly
before and after the peak, and especially for intermediate values of 
transverse momentum, we observe substantial differences between \minnlo{} and \GENEVA{} with cumulative scale choice, barely covered by scale uncertainties,
which are particularly large for \GENEVA{} in that range. Moreover, the shape
of the \GENEVA{} spectrum appears somewhat unnatural featuring a slight bump
around $70$\,GeV.

The right plot of \fig{fig:genevaptH} provides a zoomed-in view of the same observable and includes also the \GENEVA{} prediction obtained using the spectrum scale choices, as described in~\sct{sec:nutshell}, which are more appropriate for the transverse momentum spectrum. 
Moreover, in \GENEVA{}, the scale variation is implemented through variations of the profile scales at the spectrum level, making them particularly well-suited for transverse observables when using the spectrum scale choice. Indeed, the substantial 
scale uncertainties observed for the cumulant scale choice
in the matching region are strongly reduced for the spectrum scale choice, which 
features uncertainties that are of similar size as the \minnlo{} ones.
Most notably, we find that \minnlo{} and \GENEVA{} predictions are in significantly 
better agreement, when the spectrum scale choice is applied in \GENEVA{}.
In fact, the two predictions are in excellent agreement within their respective scale
uncertainties over the entire transverse momentum range.
We stress that at small transverse momenta all 5FS predictions
become essentially unphysical, as power corrections in the 
bottom mass become crucial for $p_T\lesssim m_b$, which are included only 
in 4FS calculations. Indeed, we observe that the scale uncertainty bands are
severely inflated at small transverse momenta and the predictions can
even turn negative in the first bins, because the bottom PDFs are not valid below
the bottom mass threshold.


\begin{figure}[t!]
\begin{center}
\begin{tabular}{cc}
\includegraphics[width=.45\textwidth, page=1]{plots/5fs/genevaminnlo/n3llnnloresvsMCs-withspectrum.pdf}&
\includegraphics[width=.45\textwidth, page=1]{plots/5fs/genevaminnlo/n3llan3loresvsMCs-withspectrum.pdf}
\end{tabular}
\vspace*{1ex}
\caption{Comparison of the Higgs transverse-momentum spectrum predicted by analytic resummed predictions (green, solid) at $\text{N}^3\text{LL+NNLO}$ (left plot) and at $\text{N}^3\text{LL$^{\prime}$+aN}^3\text{LO}$ against the results obtained by the two Monte Carlo generators: \minnlo{} (blue, dashed) and \GENEVA{} with cumulant (brown, dotted) and spectrum (red, solid) scale choices. \label{fig:resVSMCs}}
\end{center}
\end{figure}

We compare the transverse momentum spectrum of the Higgs boson with the resummed analytic predictions discussed in~\sct{sec:resummation}. In the left plot of~\fig{fig:resVSMCs}, we show the Monte Carlo predictions against the $\text{N}^3\text{LL}+\text{NNLO}$ result. By construction, the \GENEVA{} prediction with the spectrum scale choice matches the analytic result exactly before the parton shower, and the shower does not introduce sizable numerical effects due to the local dipole
recoil prescription selected in the shower settings of the calculations. On the other hand, the cumulant scale choice shows good agreement in the resummation region ($p_{\text{T,H}} < 20$\,GeV), but begins to substantially deviate with increasing
uncertainties at higher transverse momenta, as already observed in~\fig{fig:genevaptH}. The \minnlo{} prediction also shows good agreement with $\text{N}^3\text{LL}+\text{NNLO}$ result within its scale uncertainty band, with the largest deviations appearing at very low transverse momenta, well covered by the resummation scale variation.
In that region, $p_T\lesssim m_b$, the massless calculation actually 
breaks down and finite bottom mass effects become relevant, as pointed out before.
It therefore includes spurious effects induced by the PDF evolution, which turn 
the $\text{N}^3\text{LL}^{\prime}+\text{NNLO}$ cross section negative below $4$\,GeV, where
the figure has been cut for that reason.
In the right plot of~\fig{fig:resVSMCs}, we present the comparison of the same MCs spectra against the $\text{N}^3\text{LL$^{\prime}$+aN}^3\text{LO}$ calculation. The inclusion of higher-order corrections in the analytic spectrum results in a negative shift relative to both the \minnlo{} and \GENEVA{} predictions across the entire distribution, with a mostly flat effect for $p_{\text{T,H}} > 10$,GeV. Compared to the others, the \GENEVA{} prediction with the spectrum scale choice lies closer to the analytic result.



\begin{figure}[b!]
\begin{center}
\begin{tabular}{cc}
\includegraphics[width=.45\textwidth, page=1]{plots/5fs/genevaminnlo/minnloKQvar-geneva-yh.pdf}&
\includegraphics[width=.45\textwidth, page=1]{plots/5fs/genevaminnlo/minnloKQvar-geneva-dyhj.pdf}
\end{tabular}
\vspace*{1ex}
\caption{Higgs boson rapidity and difference in rapidity between the scalar and the leading jet as predicted by two Monte Carlo generators interfaced with \PYTHIA{8} using the \minnlo{} (blue, dashed) and \GENEVA{} (brown, dotted) methods. \label{fig:genevay}}
\end{center}
\end{figure}

We conclude the section by presenting a comparison of two angular observables in~\fig{fig:genevay}. The left plot shows the NNLO-accurate Higgs rapidity distribution, displaying ecellent agreement between the \minnlo{} prediction and the \GENEVA{} result using the cumulant scale choice, in terms of shape, normalization and size of the scale uncertainties. 
The right plot illustrates an example of an observable that depends on the leading jet. Here, we apply a flavour-blind anti-$k_T$ clustering algorithm with radius $R = 0.4$ and compute the rapidity difference between the Higgs boson and the highest-$p_T$ jet, requiring a minimum jet transverse momentum of 20\,GeV. We observe excellent agreement between the central \minnlo{} and 
\GENEVA{} predictions, with some moderate differences only in the very forward region. However, the associated scale uncertainties differ significantly between the two generators, with \GENEVA{} exhibiting a much larger theoretical uncertainty at central
rapidities.


\subsubsection{Heavy-Higgs for BSM studies in \minnlo{}}

The Higgs production in association with a bottom-quark pair is of particular interest in extensions beyond the Standard Model. Indeed, many BSM scenarios predict modifications to the Higgs-bottom quark coupling, which could lead to observable deviations in the production rates and kinematic distributions of the \bbH{} process. In many models, \bbH{} production can become the dominant production mode of exotic Higgs states. The \minnlo{} generator, introduced earlier in this section, is built within the Standard Model framework, but can be easily extended to accommodate BSM predictions. To illustrate its potential, we consider a specific example of how the NNLO+PS generator can be adapted for BSM scenarios. We specifically consider the Minimal Supersymmetric Standard Model (MSSM)~\cite{Ovrut:1984uc,Haber:1984rc,Gunion:1984yn}, which corresponds to a Type-II Two-Higgs-Doublet Model (2HDM)~\cite{Branco:2011iw} at leading order, but deviates from it at higher orders. The MSSM enforces relations between the Higgs sector and the superpartners of SM particles. Unlike the SM, the MSSM requires two Higgs doublets ($H_u$ and $H_d$) to give mass to both up-type and down-type fermions. An important parameter of the model is the ratio of vacuum expectation values ($v_u$ and $v_d$) of the two SU(2) doublets,
\begin{align}
	\tan\beta=\frac{v_u}{v_d}\,.
\end{align}
The other independent parameter is the CP-even Higgs mixing angle $\alpha$. This results in five physical Higgs bosons: two CP-even ($h$, $H$), one CP-odd ($A$) and two charged Higgs bosons ($H^{\pm}$). The lightest Higgs boson ($h$) in MSSM can mimic the SM Higgs, but has different properties depending on the model parameters. The tree-level mass of $h$ is bounded from above by $m_Z |\cos 2\beta|$, where $m_Z$ is the Z-boson mass. However, radiative corrections (mainly from the SUSY partners of the bottom and top quarks) can significantly alter the tree-level prediction, allowing for $m_h\sim 125$\,GeV~\cite{Heinemeyer:2011aa,Bechtle:2012jw,Draper:2016pys,Bechtle:2016kui,Haber:2017erd}. We perform the NNLO+PS prediction in the benchmark configuration known as $M_h^{125}$ scenario~\cite{Bagnaschi:2018ofa}, where all superparticles are chosen to be so heavy that the presence of these effects has only a mild impact on the production and decay of the light MSSM Higgs boson. Thus, the phenomenology of this scenario at the LHC closely resembles that of a Type-II 2HDM with Higgs couplings corresponding to the MSSM ones. For the chosen mass of the supersymmetric partners, we refer to eq.\,(4) of~\citere{Bagnaschi:2018ofa}. SUSY particles affect the bottom Yukawa coupling through the resummation of loop effects, which are $\tan\beta$-enhanced. We stress that the NNLO+PS calculation in the massless scheme contains only terms proportional to the squared bottom Yukawa coupling. As a result, \bbH{} predictions in the MSSM scenario differ from those in the SM solely by an overall rescaling factor. In the case of the CP-odd Higgs production, we have:
\begin{align}
	\dd \sigma_{b\bar b A}^{\text{MSSM}} = \dd \sigma_{b\bar b h}^{\text{SM}} \cdot (\tilde g_b^{A})^2\,,	\label{eq:BSMYuk}
\end{align}
with
\begin{align}
	\tilde{g}_b^A = \frac{\tan \beta}{1 + \Delta_b} \left( 1 - \Delta_b \frac{1}{\tan^2\beta} \right)\,.
\end{align}
The parameter \( \Delta_b \) resums higher-order sbottom contributions~\cite{Banks:1987iu,Hall:1993gn,Carena:1994bv,Carena:2000uj}. Electroweak corrections from neutralinos and charginos are incorporated into \( \Delta_b \) for this benchmark, with its numerical value determined by \texttt{FeynHiggs}~\cite{Heinemeyer:1998yj,Bahl:2018qog}. 
%The parameters \( g_f^\phi \) represent the genuine Yukawa couplings, which depend on the two angles governing the Higgs-matter interactions as defined in~\tab{tab:MSSMcoup}.\\
%\begin{table}[h]
%    \centering
%    \begin{tabular}{|c|c|c|}
%        \hline
%        \textbf{Coupling} & \textbf{MSSM value} \\
%        \hline
%        $g_u^H$ &  $\cos\alpha / \sin\beta$ \\
%        $g_d^H$ &  $-\sin\alpha / \cos\beta$ \\
%        $g_u^{H'}$ & $\sin\alpha / \sin\beta$ \\
%        $g_d^{H'}$ & $\cos\alpha / \cos\beta$ \\
%        $g_u^A$ & $\cot\beta$ \\
%        $g_d^A$ & $\tan\beta$ \\
%        \hline
%    \end{tabular}
%    \caption{Relative couplings $g_f^\phi$ with respect to the SM Yukawa coupling. The bottom Yukawa values correspond to the down-type couplings.}\label{tab:MSSMcoup}
%\end{table}
Following the current constraints on the $M_h^{125}$ scenario, where the lightest Higgs has a mass of $m_h= 125$\,GeV and is consistent with experimental observations, we consider the case of a CP-odd Higgs boson with a mass of 1.4 TeV and $\tan\beta = 20$, a point in the parameter space which is currently not excluded. 
We have performed the predictions by running the \minnlo{} 5FS generator with a heavy Higgs-boson mass and adjusted the Yukawa coupling according to~\eqn{eq:BSMYuk}.

\begin{figure}[t!]
\begin{center}
\begin{tabular}{cc}
\includegraphics[width=.45\textwidth, page=1]{plots/5fs/BSM/y_H-ptj30__A-1400GeV-.pdf}&
\includegraphics[width=.45\textwidth, page=1]{plots/5fs/BSM/y_H-ptj60__A-1400GeV-.pdf}
\end{tabular}
\vspace*{1ex}
\caption{Comparison of \minnlo{} results before (LHE) and after (PY8) parton shower for the rapidity distribution of the pseudo-scalar Higgs boson 
 with $m_A=1.4$ TeV and $\tan\beta=20$, including a cut on the jet transverse momentum of 30\,GeV (left) and 60\,GeV (right). \label{fig:yA}}
\end{center}
\end{figure}

\begin{figure}[t!]
\begin{center}
\begin{tabular}{cc}
\includegraphics[width=.45\textwidth, page=1]{plots/5fs/BSM/pt_Higgs__A-1400GeV-PY8-kQ0.pdf}&
\includegraphics[width=.45\textwidth, page=1]{plots/5fs/BSM/y_Higgs__A-1400GeV-PY8-kQ0.pdf}
\end{tabular}
\vspace*{1ex}
\caption{Comparison of \minlo{} and \minnlo{} results for transverse momentum and rapidity distributions of the pseudo-scale Higgs boson with $m_A=1.4$ TeV and $\tan\beta=20$. \label{fig:MiNLOBSM}}
\end{center}
\end{figure}

\begin{figure}[t!]
\begin{center}
\begin{tabular}{cc}
\includegraphics[width=.45\textwidth, page=1]{plots/5fs/BSM/pt_Higgs.pdf}&
\includegraphics[width=.45\textwidth, page=1]{plots/5fs/BSM/y_Higgs.pdf}
\end{tabular}
\vspace*{1ex}
\caption{Higgs transverse momentum and rapidity spectra for \minnlo{} results with the heavy CP-odd Higgs $\text{A}$ compared to SM predictions.\label{fig:SMvsBSM}}
\end{center}
\end{figure}

In~\fig{fig:yA}, we present a comparison of the rapidity distribution of the pseudo-scalar Higgs boson, requiring a flavour-blind anti-$k_T$ jet with radius $R = 0.4$ and 
a minimum jet transverse momentum of 30\,GeV (left plot) and 60\,GeV (right plot). We compare the results before and after interfacing the LHE events with \PYTHIA{8}. 
Since the jet spectrum becomes slightly harder after the shower, the NNLO+PS prediction features a higher rapidity distribution in the 
presence of a jet when shower effects are included. The shower effect becomes stronger for the 60\,GeV jet compared to the 30\,GeV one. 
In terms of shape, \PYTHIA{8} has a rather flat impact on the LHE predictions, although it tends to slightly increase the forward rapidity region, which becomes 
more pronounced with a higher jet transverse momentum cut. Note that we have verified that in the absence of any jet cuts, the shower does not change the 
Higgs rapidity, as expected.  

We now turn to~\fig{fig:MiNLOBSM}, where we present a comparison between \minnlo{}  and \minlo{} predictions. 
The left plot shows the transverse momentum spectrum.
Due to the high mass of the bosonic state, the resummation region extends over a broader range of the transverse momentum spectrum compared to the SM Higgs, with a mass of 125\,GeV. As a result, the NNLO corrections present in \minnlo{} have a larger impact even at intermediate transverse momentum values up to $\sim 600$\,GeV, leading to more accurate predictions and smaller scale uncertainties for \minnlo{} compared to the \minlo{} in a broad transverse momentum range. 
In the right plot of~\fig{fig:MiNLOBSM}, we compare the two predictions for the Higgs rapidity spectrum: NNLO corrections increase the cross-section with a flat correction in the central region $|y_A|<1$. Notably, the NNLO effects encoded in \minnlo{} are positive in the case of a 1.4 TeV Higgs boson, compared to the SM case where the authors of~\citere{Biello:2024vdh} have observed a flat negative correction in the Higgs rapidity spectrum.

Finally, we compare the differential behaviour of \minnlo{} predictions for the heavy Higgs state with the SM distributions in~\fig{fig:SMvsBSM}. In the left plot, we show the transverse momentum distribution. 
As expected, the BSM transverse momentum spectrum is substantially harder. With the chosen settings, BSM effects become even larger than the SM prediction for transverse-momentum values greater than 400\,GeV. 
In the right plot of~\fig{fig:SMvsBSM}, we compare the rapidity distributions. The heavy pseudo-scalar Higgs receives a larger contribution from central rapidities compared to the SM one, with the total contribution remaining always below 2\% of the SM prediction. We stress that the numerical comparison is highly sensitive to the choice of the MSSM parameters. However, the purpose of this analysis is not to focus on a specific choice of the MSSM parameters, but to demonstrate the potential of the \minnlo{} generator 
for the modelling of a BSM signal of \bbH{} production at NNLO+PS.



\subsection{NNLO+PS predictions in 4FS}\label{sec:bbH4FS}

We now turn to NNLO+PS predictions computed in the massive 4FS for the $y_{b}^{2}$ component of \bbH{} production, i.e.\ the 
corresponding 4FS calculation to the 5FS one presented in the previous section.
Section~\ref{sec:5FSNNLOPS} demonstrated that, in 5FS, the \minnlo{} and \GENEVA{} procedures provide fully exclusive 
NNLO+PS results for the \bbH{} process with massless bottom quarks. While that approach naturally resums collinear logarithms 
through the $b$-PDFs, it cannot account for power corrections in $m_{b}$, which become particularly important at 
small Higgs transverse momenta. Moreover, the 4FS is better suited to model observables that depend on the $b$-jet kinematics.

\subsubsection{\minnlo{} method for $Q\bar Q F$ production}\label{sec:bbH4FS}

We consider a fully exclusive NNLO+PS generator for the \bbH{} process in the 4FS \cite{Biello:2024pgo}.  This generator has been constructed by using 
the extension of the \minnlo{} method to heavy-quark–associated colour-singlet ($Q\bar{Q}F$) production, developed in \citere{mazzitelli:2024ura}, 
which itself builds on the \minnlo{} approach for $Q\bar{Q}$ production in \citere{mazzitelli:2020jio,mazzitelli:2021mmm}. 
While large logarithmic contributions at small transverse momentum in the $Q\bar{Q}F$ final state have the same 
general structure as for $Q\bar{Q}$ production, the more general kinematics of the $Q\bar{Q}F$ system, compared to the back-to-back configuration
for $Q\bar Q$ production, has to be accounted for in the calculation of the coefficient functions for the resummation. The singular structure is governed
by the factorisation theorem, which is expressed in Fourier-conjugate (impact-parameter or $b$) space \cite{Zhu:2012ts,Li:2013mia,Catani:2014qha,Catani:2018mei}:
\begin{align}\label{eq:facformula}
        \frac{\mathd \sigma}{ \mathd^2\vec{\pt}\, \mathd \Phi_{Q\bar{Q}{\rm F}}}&=\sum_{c=q,\bar{q},g}
  \frac{|M^{(0)}_{c\bar{c}}|^2}{2 m_{Q\bar{Q}{\rm F}}^2 }\int\frac{ \mathd^2\vec{b}}{(2\pi)^2} e^{i \vec{b}\cdot
  \vec{\pt} } e^{-S_{c\bar{c}} \left(\frac{b_0}{b}\right)}\sum_{i,j} \Tr({\mathbf H}_{c\bar{c}}{\mathbf \Delta})\,\,
  ({C}_{ci}\otimes f_i) \,({C}_{\bar{c} j}\otimes f_j) \,,
\end{align}
where $b_0=2 e^{-\gamma_\text{E}}$. The factor $e^{-S_{c\bar{c}}}$ represents the same Sudakov radiator that appears in the small-$p_{T}$ resummation for 
colour-singlet production.  In \eqn{eq:facformula}, the sum over $c=q,\bar{q},g$ spans all possible flavour assignments for the incoming partons, with first parton carrying flavour $c$ and the second parton carrying flavour $\bar{c}$. The collinear coefficient functions
$C_{ij} = C_{ij}\bigl(z,\,p_{1},\,p_{2},\,\vec{b};\,\alpha_{s}(b_{0}/b)\bigr)$ arise from collinear emissions, which include
also the constant terms, and the parton densities $f_{i}$ are evaluated at the soft scale $b_{0}/b$.  The composite factor
$\Tr\bigl({\mathbf H}_{c\bar{c}}\,{\mathbf \Delta}\bigr)\;\bigl(C_{ci}\otimes f_{i}\bigr)\;\bigl(C_{\bar{c}j}\otimes f_{j}\bigr)$
encodes the differences with respect to colour singlet production originating from the more involved colour structure 
and initial-final/final-final interferences for $Q\bar{Q}$ processes. It differs in its explicit form for the $q\bar{q}$ and $gg$ channels and written symbolically here.
 In particular, this factor contains a nontrivial Lorentz structure—omitted here for brevity—that generates azimuthal correlations in the collinear limit \cite{Catani:2010pd,Catani:2014qha}.

All quantities in bold face denote operators in colour space, and the trace $\Tr\bigl({\mathbf H}_{c\bar{c}}\,{\mathbf \Delta}\bigr)$
in \eqn{eq:facformula} runs over colour indices.  The hard function \({\mathbf H}_{c\bar{c}}\) is extracted from the infrared-subtracted amplitudes for $Q\bar{Q}F$ production, 
where the ambiguity in its definition corresponds to a choice of resummation scheme~\cite{Bozzi:2005wk}.  The operator \({\mathbf \Delta}\) encodes quantum interferences arising from soft radiation exchanged at large angles between the initial and final states, as well as among final-state quarks. It is given by ${\mathbf \Delta}={\mathbf V}^\dagger{\mathbf D}{\mathbf V}$, where
\begin{align}
\label{eq:soft}
{\mathbf V} &= {\cal
  P}\exp\left\{-\int_{b_0^2/b^2}^{m_{Q\bar{Q}F}^2}\frac{dq^2}{q^2}{\mathbf
  \Gamma}_t(\Phi_{\rm Q\bar{Q}F};\alpha_s(q))\right\}\,.%\notag\\
\end{align}
The symbol \({\cal P}\) denotes path ordering of the exponential matrix with respect to the integration variable \(q^{2}\), arranging scales from left to right in increasing order.  The anomalous dimension \(\mathbf{\Gamma}_{t}\) governs the effect of real soft radiation emitted at large angles.  Meanwhile, the azimuthal operator $\mathbf{D} \;\equiv\; \mathbf{D}\bigl(\Phi_{Q\bar{Q}F},\,\vec{b},\,\alpha_{s}\bigr)$
encodes azimuthal correlations of the \(Q\bar{Q}F\) system in the small-\(p_{T}\) limit.  Upon averaging over the azimuthal angle \(\phi\), it satisfies
$\bigl[\mathbf{D}\bigr]_{\phi} \;=\; \mathbf{1}$.
Starting from the $Q\bar Q F$ resummation formula, the procedure to construct a \minnlo{} improved $\bar{B}$ function system is the same as that of the colour-singlet case.
All technical details of this derivation are provided in Refs.~\cite{mazzitelli:2020jio,mazzitelli:2021mmm,mazzitelli:2024ura,Biello:2024pgo}.

In the case of \bbH{} production in the 4FS, the two-loop virtual corrections, required for NNLO accuracy,
are not known in exact form with finite $m_b$. Instead, the two-loop amplitude is approximated in the NNLO+PS
prediction by a small-mass expansion $m_b$, i.e.\ all logarithmically enhanced terms and constant terms are retained,
while dropping terms that are power-suppressed $m_b$. 
This massification procedure, developed in \citeres{Mitov:2006xs,Wang:2023qbf}, allows us to capture the dominant 
virtual corrections in the small-$m_b$ limit. The massless full-colour two-loop amplitudes, which provide the constant
terms, are taken from~\citere{Badger:2024mir}. Different from the studies in \citere{Biello:2024pgo}, which relied on 
the leading-colour approximation~\cite{badger:2021ega}, here we perform a phenomenological analysis using events 
generated with the full-colour library of~\citere{Badger:2024mir} for the two-loop constant terms in the massification. 
The NNLO calculation is thus complete in full colour, up to power-suppressed in the bottom mass, which are neglected
only in the two-loop amplitude.

\subsubsection{Results and flavour-scheme comparison}
The setup for the phenomenological results presented here—including input parameters, renormalisation and factorisation scale choices and PDFs —follows the conventions outlined in \sct{sec:setup}. We now compare theoretical predictions obtained in the 5FS and 4FS for Higgs production in association with bottom quarks. The results are organised in two stages: first, we consider fully inclusive observables that do not rely on identifying bottom quarks in the final state; subsequently, we turn to observables involving b-jets, with a careful definition of flavour-tagged jets for a meaningful comparison between schemes.
\begin{table}[ht!]
  \vspace*{0.3ex}
  \begin{center}
	   \renewcommand{\arraystretch}{1.6}
    \begin{tabular}{|c||c|c|c|c|}
    \hline
    \makecell[c]{\shortstack{\rule{0pt}{2ex}Fiducial region}} &  
    \makecell[c]{\shortstack{\rule{0pt}{2ex}NLO$_{\rm PS}$ \\ 5FS} } & 
    \makecell[c]{\shortstack{\rule{0pt}{2ex}NLO$_{\rm PS}$ \\ 4FS} }  & 
    \makecell[c]{\shortstack{\rule{0pt}{2ex}\minnlo{} \\ 5FS} } &  
    \makecell[c]{\shortstack{\rule{0pt}{2ex}\minnlo{} \\ 4FS ($a2\ell_{\rm FC}$)\footnotemark} } \\
    \hline \hline
	    H & $725.4(2)_{-10\%}^{+11\%}$ & $ 389.0(1)_{-20\%}^{+24\%}$ & $ 574.8(4)_{-8.0\%}^{+4.5\%}$ & $ 519.6(3)_{-15\%}^{+19\%}$\\
     \hline
	    H $+\geq1\,bj_{\text{IFN}}$ & $104.7(1)_{-10\%}^{+10\%}$ & $ 76.44(6)_{-20\%}^{+26\%}$ & $ 115.9(1)_{-8.8\%}^{+9.3\%}$& $ 98.57(2)_{-12\%}^{+7.7\%}$\\
      \hline
	    H $+\geq2\,bj_{\text{IFN}}$ & $5.911(3)_{-10\%}^{+10\%} $ & $ 5.001(1)_{-23\%}^{+32\%}$ & $ 8.464(3)_{-10\%}^{+11\%}$& $ 7.028(6)_{-10\%}^{+1.6\%}$ \\
       \hline
        H $+0\,bj_{\text{IFN}}$  & $620.7(9)_{-10\%}^{+11\%}$ & $ 312.6(4)_{-20\%}^{+23\%}$ & $ 458.9(3)_{-7.8\%}^{+3.3\%}$&$ 421.0(2)_{-15\%}^{+21\%}$ \\
        \hline
    \end{tabular}
  \end{center}
  \vspace{-1em}
  \caption{
	Cross-section values in fb of the different \POWHEG{} generators in massless and massive schemes. 
	\label{tab:NNLO4FS_xs}}
\end{table}
\footnotetext{%
  ($a2\ell_{\rm FC}$) stands for approximate two-loop amplitudes, obtained by massifying the massless amplitude in the full-colour limit.
}
\begin{figure}[t!]
\begin{center}
\begin{tabular}{cc}
\includegraphics[width=.45\textwidth, page=1]{plots/4fs/pt_Higgs_NLO_5FS_4FS.pdf}&
\includegraphics[width=.45\textwidth, page=1]{plots/4fs/pt_Higgs_minnlops_5FS_4FS-FC.pdf}
\end{tabular}
\vspace*{1ex}
\caption{Comparison between different flavour scheme choices for the Higgs transverse momentum spectrum at NLO+PS (left) and NNLO+PS level (right). \label{fig:4fsA}}
\end{center}
\end{figure}
\subsubsection*{Predictions for Higgs observables}
In the 4FS, NNLO corrections increase the total cross section by about 30\% relative to NLO, as shown in \tab{tab:NNLO4FS_xs}. This highlights the importance of NNLO accuracy for precise predictions in the massive scheme. The \minnlo{} result in the 4FS shows reduced dependence on the scale choices as compared to the NLO+PS prediction. However, \minnlo{} prediction in the 5FS—based on the process \(b\bar{b} \to H+X\)—has a smaller scale uncertainties. 
%This difference arises from the structure of the perturbative expansion: in the 4FS, strong couplings appear already at Born level, and the NNLO corrections are relatively large.
Despite these differences, the NNLO+PS predictions in both schemes agree well within their respective scale uncertainties, demonstrating that the long-standing tension between the 4FS and 5FS inclusive results is reduced once NNLO accuracy is included.
\begin{figure}[t!]
\begin{center}
\begin{tabular}{cc}
\includegraphics[width=.45\textwidth, page=1]{plots/4fs/pt_H-IFN-1bjet_minnlops-FC.pdf}&
\includegraphics[width=.45\textwidth, page=1]{plots/4fs/pt_H-IFN-2bjet_minnlops-FC.pdf}
\end{tabular}
\vspace*{1ex}
\caption{Higgs transverse momentum spectrum with at least one (left plot) or two (right plot) $b$-jets identified using the IFN algorithm applied to the events generated by \minnlo{} in the massless (blue, dashed) and massive (red, solid) schemes.\label{fig:4fsB}}
\end{center}
\end{figure}

Further insight comes from the Higgs transverse-momentum distribution, shown in figure~\ref{fig:4fsA}. At NLO+PS, the 4FS and 5FS predictions differ significantly at low \(p_{T,H}\). At NNLO+PS, this discrepancy is significantly reduced: \minnlo{} matching brings the two predictions into much better agreement across the full \(p_{T,H}\) range. In particular, in the hard region with \(p_{T,H} \gtrsim 75\)~GeV, the 4FS prediction lies well within the scale uncertainty band of the 5FS prediction.
 
\subsubsection*{Predictions with $b$-jet tagging}
We now turn to the comparison of results with at least one $b$-jet identified in the final state. In this context, the definition of $b$-jets becomes particularly important, especially in the massless 5FS. Applying a jet-flavour algorithm to fixed-order predictions in a massless scheme introduces potential issues due to a mismatch between real and virtual contributions. These mismatches can lead to infrared-unsafe predictions for observables involving flavoured jets, particularly in configurations where a gluon splits into a $b\bar{b}$ pair within the same jet, or when soft, wide-angle $b$-quark emissions are clustered into a jet containing a hard parton. 

%Such divergences arise in tagging strategies like the experimental $b$-tagging definition, referred to as \texttt{EXP} hereafter, which can, however, be directly implemented in the massive 4FS predictions. 
To ensure a theoretically sound and IRC-safe definition of $b$-jets in the massless scheme, we employ the \emph{Interleaved Flavour Neutralisation} (IFN) algorithm~\cite{Caola:2023wpj}, which systematically removes soft $b$-flavour contamination during clustering. In our setup, we use the recommended parameters $\alpha = 2$ and $\omega = 1$. 
%The IFN algorithm is implemented through a dedicated Fortran--C++ interface to the \texttt{FastJet} plugin, allowing seamless integration into our \POWHEG{} framework for all relevant processes.

We now examine the fiducial cross sections in the presence of $b$-jets, comparing 4FS and 5FS predictions at both NLO+PS and NNLO+PS accuracy. As shown in \tab{tab:NNLO4FS_xs}, requiring at least one $b$-jet leads to a significant reduction in the inclusive cross section, by a factor of approximately 5. Imposing a second $b$-jet further suppresses the rate by roughly an additional order of magnitude. These trends are seen across both schemes and perturbative orders. At NNLO+PS, the inclusion of higher-order corrections via \minnlo{} improves the overall consistency between the two schemes, particularly in the one- and two-$b$-jet bins, where 4FS results remain slightly below the 5FS, but still agree within uncertainties.

% Interestingly, we observe that the NLO+PS predictions in 4FS and 5FS for the $\geq$2 $b$-jet bin are numerically very close—more so than at NNLO+PS. However, this agreement is accidental. In the 5FS, such observables are less reliable at NLO, since bottom quarks only arise from gluon splittings and are not part of the hard matrix element at Born level. The good agreement at the integrated cross-section level does not reflect the underlying physics accuracy, as will become evident in the following, where we examine the corresponding differential distributions.

Figure~\ref{fig:4fsB} shows the Higgs transverse momentum distribution in events with at least one (left) or two (right) $b$-jets identified using the IFN tagging algorithm. For the inclusive one-$b$-jet selection, the differences between the schemes remain moderate and largely flat. In the two-$b$-jet case, the 4FS prediction is systematically lower than the 5FS, especially in the low and intermediate $p_{T,H}$ regions. This behaviour reflects the more accurate treatment of bottom-quark kinematics in the massive scheme, which becomes particularly relevant when both $b$-jets are required to be hard and well separated.

These results underline the importance of the 4FS computation for observables involving identified $b$-jets. Due to the partonic structure of the 5FS calculation, it lacks accuracy in the hard matrix-element description of $b$-jet production compared to the 4FS. As a result, the 4FS provides a more reliable and accurate prediction, especially in regions sensitive to the kinematics and multiplicity of bottom jets. This makes the 4FS prediction highly relevant for comparisons with experimental measurements involving $b$-tagged final states.

%%%% starting HH chapter

\section{Modelling of the \boldmath{$b\bar bH$} background in HH searches}\label{sec:HH}


%In this section we present rates for the $y_t^2$ contribution, focusing in particular on the signal region of interest di-Higgs ($HH$) searches~\cite{}. This closely follows what has been presented in \citere{manzoni:2023qaf}. At variance with that work, we update the collider energy and the input parameter in par with the HWG recommendations. We preesent predictions in the 4FS at NLO
%QCD accuracy matched with parton shower, specifically with {\sc Pythia8}~\cite{Bierlich:2022pfr}. Results are obtained in an automatic way with
%{\sc MadGraph5\_aMC@NLO}~\cite{Alwall:2014hca,Frederix:2018nkq}. We compare these results with the inclusive gluon-fusion {\sc NNLOPS} simulation in the 5FS
%{\bf MZ cite ad comment}.
%
%The HH signal region is inspired to the $bb\gamma\gamma$ final state. In particular, we require two $b$-tagged, anti-$k_T$~\cite{Cacciari:2005hq,Cacciari:2008gp,Cacciari:2011ma} jets with
%\begin{equation}
%\label{eq:ptjcuts}
%p_T(j) > 25,\GeV \qquad {\rm and} \qquad |\eta(j)| < 2.5\,,
%\end{equation}
%and with their invariant mass compatible with the Higgs
%\begin{equation}
%    \label{eq:mbjcut}
%    80\, \GeV < m(b_1,b_2)< 140\,\GeV\,.
%\end{equation}
%Furthermore, we require the presence of two photons (in our simulations, they are produced by the Higgs decay), which satisfy the following cuts:
%\begin{equation}
%\label{eq:photoncuts}
%    105\,\GeV < m(\gamma_1, \gamma_2) < 160\,\GeV, \quad |\eta(\gamma_i)|< 2.37, \quad
%    \frac{p_T(\gamma_1)}{m(\gamma_1, \gamma_2)} > 0.35, \quad \frac{p_T(\gamma_2)}{m(\gamma_1, \gamma_2)} > 0.25\,.
%\end{equation}
%The presence of two $b$ jets and two photons passing the aforementioned cuts defines the \emph{fiducial} region. Such a selection
%can be made more strict by employing the kinematic variable
%\begin{equation}
%\label{eq:mbbggs}
%    \mbbggs = m_{2b2\gamma} - m(b_1,b_2) - m(\gamma_1,\gamma_2) + 2m_H \,,
%\end{equation}
%and further imposing cuts on it (see the ATLAS analysis \cite{ } {\bf MZ add citation!!!}). We will consider two scenarios 
%where the fiducial cuts are supplemented by the requirement $\mbbggs< 500$ GeV or  $\mbbggs< 350$ GeV. Our results are shown in
%Tab.~\ref{tab:XS-fid}. Numbers include the $H\to \gamma\gamma$ branching ratio, $\textrm{BR}(H\to\gamma\gamma)=0.227\%$~\cite{LHCHiggsCrossSectionWorkingGroup:2016ypw}.
%We quote the uncertainties stemming from scale variations, where the factorisation and renormalisation scale are varied independently around 
%the central value by a factor 2 (9-point variatons). Other sources of uncertainties, such as those stemming from the variation
%of the shower starting scale, from the usage of different parton showers, or from variations of the MC@NLO matching details~\cite{Frixione:2002ik,Frederix:2020trv}
%have been considered in \citere{manzoni:2023qaf} and have been found to be subleading.
%
%{\bf MZ part on minnlops to be written by S. Manzoni}
%Focus on \citere{manzoni:2023qaf} mainly (and \citere{Biello:2024pgo}) in fiducial cuts for HH searches (ref. Stefano Manzoni). NNLO+PS ggF generator with different $n_f$ is running using ATLAS resources.



%In this section, we present rate predictions for the $y_t^2$ contribution, focusing in particular on the signal region relevant for in Higgs boson pair production ($HH$) searches~\cite{}. Our study closely follows the work presented in \citere{manzoni:2023qaf}, with the main differences being an updated collider energy and input parameters in line with setup discussed in section~\ref{}. We provide predictions in the 4FS at NLO QCD accuracy, matched to PS using {\sc Pythia8}~\cite{Bierlich:2022pfr}. The results are obtained automatically with {\sc MadGraph5\_aMC@NLO}~\cite{Alwall:2014hca,Frederix:2018nkq}. We compare these results to an inclusive gluon-fusion NNLOPS~\cite{Hamilton:2012rf,Hamilton:2013fea,Hamilton:2015nsa} simulation in the 5FS, generated using {\sc Powheg\_Box\_v2}~\cite{nason:2004rx,frixione:2007vw,alioli:2010xd}. This setup achieves NNLO accuracy for inclusive Higgs boson observables. However, it only provides LO accuracy for final states in which the Higgs boson is produced in association with two additional jets. In the {\sc NNLOPS} calculation the renormalisation and factorisation scales are set to $\muR = \muF = m_H/2$, the PDF4LHC21 set for the parton distribution functions are used~\cite{PDF4LHCWorkingGroup:2022cjn},
%and Pythia8 is employed to perform the parton showering and to include the decay of the Higgs boson to two photons.

In this section, we present rate predictions for the $y_t^2$ contribution, focusing in particular on the signal region relevant for Higgs boson pair production ($HH$) searches~\cite{HDBS-2021-18,CMS-HIG-22-001}. Our study closely follows the work presented in \citere{manzoni:2023qaf}, with the main differences being the updated collider energy and input parameters, as described in section~\ref{sec:setup}. Predictions are provided in the 4FS at NLO QCD accuracy, matched to the \PYTHIA{8}~\cite{Bierlich:2022pfr} parton shower. The results are obtained automatically with {\sc MadGraph5\_aMC@NLO}~\cite{Alwall:2014hca,Frederix:2018nkq}.

We compare these predictions to those from an inclusive gluon-fusion {\sc NNLOPS} simulation~\cite{hamilton:2012rf,hamilton:2013fea,hamilton:2015nsa} in the 5FS, generated using {\sc Powheg\_Box\_v2}~\cite{nason:2004rx,frixione:2007vw,alioli:2010xd}. This setup achieves NNLO accuracy for inclusive Higgs boson observables, but only provides LO accuracy for final states in which the Higgs boson is produced in association with two additional jets. In the {\sc NNLOPS} calculation, the renormalisation and factorisation scales are set to $\mu_R = \mu_F = m_H/2$, the PDF4LHC21 parton distribution functions are used~\cite{PDF4LHCWorkingGroup:2022cjn}, and \PYTHIA{8} is employed both for parton showering and for simulating the decay of the Higgs boson into two photons.



The HH signal region is inspired by the $HH \to b\bar{b}\gamma\gamma$ final state~\cite{HDBS-2021-10,CMS-HIG-19-018}. Specifically, we require two $b$-tagged anti-$k_T$~\cite{Cacciari:2005hq,Cacciari:2008gp,Cacciari:2011ma} jets satisfying
\begin{equation}
\label{eq:ptjcuts}
p_T(j) > 25\,\GeV \qquad {\rm and} \qquad |\eta(j)| < 2.5\,,
\end{equation}
with their invariant mass constrained to be compatible with the Higgs boson mass:
\begin{equation}
    \label{eq:mbjcut}
    80\,\GeV < m(b_1,b_2)< 140\,\GeV\,.
\end{equation}
Additionally, we require two photons—produced in the simulation via Higgs decays—to pass the following selection criteria:
\begin{equation}
\label{eq:photoncuts}
    105\,\GeV < m(\gamma_1, \gamma_2) < 160\,\GeV, \quad |\eta(\gamma_i)|< 2.37, \quad
    \frac{p_T(\gamma_1)}{m(\gamma_1, \gamma_2)} > 0.35, \quad \frac{p_T(\gamma_2)}{m(\gamma_1, \gamma_2)} > 0.25\,.
\end{equation}
The presence of two $b$ jets and two photons satisfying the above requirements defines the \emph{fiducial} region. This selection can be further refined in the context of $HH$ analyses by imposing a cut on the variable
\begin{equation}
\label{eq:mbbggs}
    \mbbggs = m_{2b2\gamma} - m(b_1,b_2) - m(\gamma_1,\gamma_2) + 2m_H \,.
\end{equation}
This observable is used in $HH$ searches~\cite{HDBS-2021-10,HDBS-2019-27,CMS-HIG-19-018,ATLAS:2025hhd} to separate the selected events into two regions of phase space: one that is more sensitive to potential modifications of the Higgs self-coupling ($m_{2b2\gamma}^{*} < 350$~GeV), and one that is more sensitive to the Standard Model production ($m_{2b2\gamma}^{*} > 350$~GeV). We consider two scenarios in which the fiducial selection includes the requirements $\mbbggs < 500$\,GeV and $\mbbggs < 350$\,GeV, respectively. The numerical results comparing the \bbH{} $y_t^2$ component, computed at NLO in the 4FS using {\sc MadGraph5\_aMC@NLO}, with the expected rate obtained from the {\sc NNLOPS} simulation, with and without $g\rightarrow b\bar{b}$ splitting included in the parton shower, are presented in table~\ref{tab:XS-fid}. For reference, table~\ref{tab:XS-fid} also includes the rates in the same fiducial regions for the \bbH{} $y_b^2$ component, evaluated at NNLO accuracy as discussed in section~\ref{sec:bbH4FS}. All rates include the Higgs branching ratio into photons, $\textrm{BR}(H \to \gamma\gamma) = 0.227\%$~\cite{LHCHiggsCrossSectionWorkingGroup:2016ypw}. Uncertainties due to renormalisation and factorisation scale variations are quoted, where each scale is independently varied by a factor of two around the central value following the standard 9-point prescription. Additional theoretical uncertainties such as those associated with variations of the shower starting scale, the choice of parton shower algorithm, or modifications to the MC@NLO matching procedure~\cite{frixione:2002ik,frederix:2020trv} were examined in \citere{manzoni:2023qaf} and found to be subleading. {\sc NNLOPS} uncertainty
are evaluated via multiple sources accounting for the modelling of jet multiplicity, the Higgs boson $\ptH$, $\ptHjj$, and $\mjj$~\cite{deFlorian:2016spz,Liu:2013hba,stewart:2013faa,Boughezal:2013oha,Stewart:2011cf,Gangal:2013nxa}, as discussed in \citere{ATLAS:2022tnm}. Parton shower uncertanties are also neglected for the {\sc NNLOPS} prediction, but they could be sizeable. 
%---------------Table------------------------------------------------------------------------------------------
\begin{table}[ht!]
  \vspace*{0.3ex}
  \begin{center}
	   \renewcommand{\arraystretch}{1.2}
\begin{tabular}{|c|c|c|c|c|}
  \hline 
	Fiducial region  & \makecell[c]{\shortstack{\rule{0pt}{2ex} $y_b^2$ NNLO   \\ (\bbH{} \minnlo{})} }  &  \makecell[c]{\shortstack{\rule{0pt}{2ex} $y_t^2$ NLO   \\ (NLO+PS)} } & \makecell[c]{\shortstack{\rule{0pt}{2ex} $y_t^2$ LO  \\ (ggF NNLOPS)} }  & \makecell[c]{\shortstack{\rule{0pt}{2ex} $y_t^2$ LO  $\cancel{g\to b\bar{b}}$  \\ (ggF NNLOPS)} }   \\
  \hline \hline
   \begin{tabular}{@{}c@{}} No cut \end{tabular}
%  \multirow{1}{*}{\makecell{No cut}} 
	   & $1180_{-15\%}^{+19\%}$ & $1696 ^{+62\%} _{-35\%}$  & $000_{-00\%}^{+00\%}$\\
  \hline
   \begin{tabular}{@{}c@{}} Fid. cuts \end{tabular}
 % \multirow{1}{*}{\makecell{Fid. cuts}} 
	   &$5.60_{-10\%}^{+1.9\%}$ & $18 ^{+55\%} _{-33\%}$ & $000_{-00\%}^{+00\%}$ \\
   \hline
  \begin{tabular}{@{}c@{}} Fid. cuts \\+ $\mbbggs<500\,\GeV$  \end{tabular}
  %\multirow{1}{*}{\makecell{Fid. cuts \\+ $\mbbggs<500\,\GeV$}} 
	  & $5.5_{-10\%}^{+2.0\%}$ & $ 12 ^{+57\%} _{-33\%}$ & $000_{-00\%}^{+00\%}$\\
   \hline
    \begin{tabular}{@{}c@{}} Fid. cuts \\+ $\mbbggs<350\,\GeV$  \end{tabular}
    %\multirow{1}{*}{\makecell{Fid. cuts \\+ $\mbbggs<350\,\GeV$}} 
    & $4.8_{-10\%}^{+2.8\%}$ & $5.5 ^{+60\%} _{-34\%}$ & $000_{-00\%}^{+00\%}$ \\
   \hline
\end{tabular}
  \end{center}
  \vspace{-1em}
  \caption{
Cross sections in ab for the $y_b^2$ and $y_t^2$ contributions to $pp \to b \bar b H$ with $H\to \gamma\gamma$ decay at $\sqrt{s}=13.6$ TeV. \label{tab:XS-fid}}
\end{table}


Significant differences are observed between the {\sc MadGraph5\_aMC@NLO} and {\sc NNLOPS} predictions with the {\sc NNLOPS} fiducial cross section nearly twice as large when the  $g\rightarrow b\bar{b}$ splitting is included in the parton shower. As discussed in~\citere{manzoni:2023qaf}, the parton shower likely induces a double counting of the gg$H$ + 2~$b$-jets process, which is already included at matrix-element level in the {\sc NNLOPS} generation and is formally beyond the nominal accuracy of that prediction. The mismodelling originated by the {\sc NNLOPS} sample is currently covered by a conservative 100\% uncertainty in $HH$ analyses~\cite{HDBS-2021-10,ATLAS:2025hhd}. 

In all categories, the $y_b^2$ and $y_t^2$ contributions to the \bbH{} cross section remain of the same order as the $HH$ signal cross section~\cite{manzoni:2023qaf} (not included in the table), highlighting the importance of their accurate modelling to obtain a precise measurement of the $HH$ process. The $y_t^2$ term dominates the \bbH{} cross section in most of the fiducial regions, except in the lower $\mbbggs$ phase space, where its contribution becomes comparable to $y_b^2$ one. As shown in \fig{fig:4fsMBB} and \fig{fig:4fsFID}, the inclusion of NNLO corrections increases the $y_b^2$ contribution by up to 60\%-40\% and leads to a significant reduction in scale uncertainties. In \fig{fig:4fsNNLOPS}~\citere{atlaspub}, the distributions of $p_T^H$ and $\mbbggs$ are presented for the {\sc MadGraph5\_aMC@NLO} and {\sc NNLOPS} samples, after applying the $b$-jet-only selection defined in \eqref{eq:ptjcuts}. The distributions are normalised to unity to highlight the harder spectra predicted by {\sc MadGraph5\_aMC@NLO}, when disentangled from the large rate differences observed with respect to {\sc NNLOPS}.






\begin{figure}[t!]
\begin{center}
\begin{tabular}{cc}
\includegraphics[width=.45\textwidth, page=1]{plots/4fs/n_b_jets-EXP.pdf}&
\includegraphics[width=.45\textwidth, page=1]{plots/4fs/m_bb-EXP-2bjet.pdf}
\end{tabular}
\vspace*{1ex}
\caption{The left plot represents the number of \texttt{EXP} $b$-jets in the NLO+PS and NNLO+PS bottom-Yukawa-induced events. The right plot shows the impact of NNLO corrections in the invariant-mass spectrum of the leading two b-jets.\label{fig:4fsMBB}}
\end{center}
\end{figure}

\begin{figure}[t!]
\begin{center}
\begin{tabular}{cc}
\includegraphics[width=.45\textwidth, page=1]{plots/4fs/pt_Higgs-EXP-fid-FC.pdf}&
\includegraphics[width=.45\textwidth, page=1]{plots/4fs/mass_2b2gam-EXP-fid-FC.pdf}
\end{tabular}
\vspace*{1ex}
\caption{Transverse-momentum spectrum of the Higgs boson and invariant mass $\mbbggs$ as defined in~\eqn{eq:mbbggs} in the fiducial volume described by the cuts~(\ref{eq:ptjcuts}-\ref{eq:photoncuts}).\label{fig:4fsFID}}
\end{center}
\end{figure}


\begin{figure}[t!]
\begin{center}
\begin{tabular}{cc}
\includegraphics[width=.45\textwidth, page=1]{plots/ATLAS/BJetCuts_Higgs1_Pt_shape_comparison_ratio.pdf}&
\includegraphics[width=.45\textwidth, page=1]{plots/ATLAS/BJetCuts_yybb_Mass_shape_comparison_ratio.pdf}
\end{tabular}
\vspace*{1ex}
\caption{Distributions~\cite{atlaspub} of the Higgs boson $\pt$, and invariant mass $\mbbggs$ as defined in~\eqn{eq:mbbggs}  for the {\sc MadGraph5\_aMC@NLO} and {\sc NNLOPS} samples after requiring at least 2 $b$-jets in the final state as describe by \eqref{eq:ptjcuts}. All distributions are normalised to unity. The blue and green shaded bands represent the scale uncertainties (shape only) of the {\sc NNLOPS} and {\sc MadGraph5\_aMC@NLO} samples, respectively. The lower panel displays the ratio of the two predictions.\label{fig:4fsNNLOPS}}
\end{center}
\end{figure}

\subsection{Combination of 4FS $y_t^2$ contribution with {\sc POWHEG NNLOPS}}

In analyses focusing on final states containing $b$-jets, it is essential to accurately model the contributions from light-flavor and charm jets that may enter the event selection due to mis-tagging. To facilitate the use of the {\sc MadGraph5\_aMC@NLO} sample in experimental physics studies while maintaining a realistic description of all jet flavour components, the $gg \to b\bar{b}H$ process is supplemented with the {\sc NNLOPS} $H + j$ sample. The {\sc NNLOPS} simulation includes the $gg \to H$ process accompanied by light-flavor and charm jets, which are not represented in the {\sc MadGraph5\_aMC@NLO} sample.

To prevent double counting of events featuring $gg \to H$ production with two $b$-jets, the {\sc NNLOPS} sample is filtered to exclude any event that contains at least one bottom quark, whether originating from the hard scatter or from parton showering. To ensure complete orthogonality between the samples, events with a single bottom quark need also to be removed from the {\sc NNLOPS} sample. This excludes $g + b \to b + H$ contributions, which are already accounted for in the {\sc MadGraph5\_aMC@NLO} prediction via unresolved $g \to b\bar{b}$ splittings. This effectively removes the $b\bar{b}H$ component from {\sc NNLOPS} sample. The filtered {\sc NNLOPS} sample can then be reliably combined with the {\sc MadGraph5\_aMC@NLO} prediction for $gg \to b\bar{b}H$. Although this method involves combining results from the 5FS and 4FS, it provides a coherent and practical approach for estimating the $gg \to H$ plus $b$-jets contribution at analysis level. This strategy is consistent with approaches used in similar contexts, such as the modelling of $Z$ boson production with two bottom quarks in Ref.~\cite{bagnaschi:2018dnh}, and the treatment of the $t\bar{t}+b\bar{b}$ process in~\citere{PhysRevD.93.014019}.  As discussed in~\citere{}, the filtering efficiency, defined as the fraction of {\sc NNLOPS} events surviving the bottom quark veto, is found to be 94\%. The filtered {\sc NNLOPS} sample is then normalized to the inclusive $gg \to H$ cross section at N$^3$LO, after subtracting $\sigma_{gg \to \bbH}^{\text{NLO}}$ for the $y_t^2$ component as listed in~\tab{tab:bbHyt13p6}~and~\tab{tab:bbHyt13}. The comparison of nominal and filtered {\sc NNLOPS} samples is discussed in~\citere{atlaspub}.


\begin{table}[ht!]
\begin{center}%
\begin{small}%
\tabcolsep5pt
\begin{tabular}{cccc}%
$m_h$[GeV] & $\sigma^{}$[fb] & $\Delta_{\left(\mu_{R},\mu_{F}\right)\oplus\mu_{B}}$[\%] & $\Delta_{\mathrm{PDF}\oplus\alpha_s}$[\%]  \\\hline
$125.00$  \\
$125.09$ \\
$125.10$  \\
$125.20$ \\
\end{tabular}%
\end{small}%
\end{center}%
\caption{Total \bbH{} cross sections for the $y_t^2$ contribution in the SM for a LHC CM energy of $\sqrt{s}=13.6$ TeV.}
\label{tab:bbHyt13p6}
\end{table}


\begin{table}[ht!]
\begin{center}%
\begin{small}%
\tabcolsep5pt
\begin{tabular}{cccc}%
$m_h$[GeV] & $\sigma^{}$[fb] & $\Delta_{\left(\mu_{R},\mu_{F}\right)\oplus\mu_{B}}$[\%] & $\Delta_{\mathrm{PDF}\oplus\alpha_s}$[\%]  \\\hline
$125.00$ \\
$125.09$\\
$125.10$  \\
$125.20$ \\
\end{tabular}%
\end{small}%
\end{center}%
\caption{Total \bbH{} cross sections for the $y_t^2$ contribution in the SM for a LHC CM energy of $\sqrt{s}=13$ TeV.}
\label{tab:bbHyt13}
\end{table}




\begin{figure}[t!]
\begin{center}
\begin{tabular}{cc}
%\includegraphics[width=.45\textwidth, page=1]{plots/ATLAS/BJetCuts_Higgs1_Pt_shape_comparison_ratio.pdf}&
%\includegraphics[width=.45\textwidth, page=1]{plots/ATLAS/BJetCuts_yybb_Mass_shape_comparison_ratio.pdf}
\end{tabular}
\vspace*{1ex}
\caption{Distributions~\citere{atlaspub} of the Higgs boson $\pt$, and invariant mass $\mbbggs$ as defined in~\eqn{eq:mbbggs}  for the {\sc MadGraph5\_aMC@NLO} and {\sc NNLOPS} samples after requiring at least 2 $b$-jets in the final state as describe by \eqref{eq:ptjcuts}. All distributions are normalised to unity. The blue and green shaded bands represent the scale uncertainties (shape only) of the {\sc NNLOPS} and {\sc MadGraph5\_aMC@NLO} samples respectively. The lower panel displays the ratio of the two predictions.\label{fig:4fsNNLOPS}}
\end{center}
\end{figure}




\section{Light-quark Yukawa contributions to Higgs production}\label{sec:lightYukawa}
In addition to the $b\bar{b}H$ channel, studies of light-quark fusion are also interesting due to their ability to constrain the light-quark Yukawa couplings. Indeed, while the bottom-quark Yukawa coupling is strongly constrained through measurements of Higgs decays, no stringent bounds exist for lighter quarks~\cite{Kagan:2014ila}. In particular, the charm Yukawa coupling is only weakly constrained, with an observed upper limit of less than 8.5 times the Standard Model prediction based on analyses of Higgs decay products in Higgsstrahlung events~\cite{Atlas:2022ers}. In ref.~\cite{Bishara:2016jga} it was suggested that bounds could be obtained from measurements of the Higgs transverse momentum spectrum, which could be used to measure the bottom and charm Yukawa couplings and  enable upper bounds to be placed on couplings of lighter quarks. The cross-section for single Higgs production via parton fusion is given by:
\begin{align}
\sigma_{\text{H}}(\bar \kappa_q^2)=\sigma_{b\bar b \rightarrow \text{H}}+\bar \kappa_c^2 \bar \sigma_{c\bar c \rightarrow \text{H}}+\bar \kappa_s^2 \bar \sigma_{s\bar s \rightarrow \text{H}}+\bar \kappa_d^2 \bar \sigma_{d\bar d \rightarrow \text{H}}+\bar \kappa_u^2 \bar \sigma_{u\bar u \rightarrow \text{H}}\,.
\end{align}
All interference effects at higher orders vanish due to the massless treatment of the \( n_f = 5 \) parton species. In this context, we introduce the coupling \( \bar\kappa_q \), which denotes the Yukawa coupling of a light quark \( q \), normalised to the Standard-Model Yukawa coupling of the bottom quark in the \MSbar{} scheme. In other words, they represent the cross sections obtained by using \( q \)-flavoured PDFs and assigning the Yukawa interaction the strength of the bottom-quark coupling.

A value of \( \bar\kappa_q = 1 \) corresponds to a Yukawa coupling for quark \( q \) equal in strength to that of the bottom quark. While the SM expectation is \( \kappa_q = 1 \), the normalised couplings take the following values in the SM:  
\( \bar \kappa_u \simeq 5.17 \cdot 10^{-4} \),  
\( \bar \kappa_d \simeq 1.12 \cdot 10^{-3} \),  
\( \bar \kappa_s \simeq 2.2 \cdot 10^{-2} \).
The High-Luminosity Large Hadron Collider prospects~\cite{deBlas:2019rxi} shows that the light-quark Yukawa coupling can be constrained to $|\bar \kappa_u|\leq 0.66$, $|\bar \kappa_d|\leq 0.70$, $|\bar \kappa_s|\leq 0.66$ at 95\% CL when using the transverse momentum distribution.

In ref.~\cite{Cal:2023mib}, resummed predictions for the Higgs transverse momentum spectrum were obtained for $c\bar{c}H$ and $s\bar{s}H$ channels at N$^3$LL$^{\prime}$+aN$^3$LO accuracy. These resummed calculations (at a lower, N$^3$LL accuracy) were used to construct a \textsc{Geneva} NNLO+PS generator for the $c\bar{c}H$ process in ref.~\cite{Gavardi:2025zpf} which combines the advantages of having a high resummation accuracy with a fully differential event generator.

In addition, light-quark fusion has recently been implemented in the fully-differential \minnlo{} generator. In this paper, we present the novel NNLO+PS simulations with a detailed phenomenological study for Higgs-boson production with diphoton decay, $q\bar q\rightarrow \text{H}\rightarrow \gamma\gamma$. The simulations are produced using the \minnlo{} \bbH{} generator in the massless scheme as a starting point. The code modifies the flavour of the initial-state partons via the \texttt{whichlightquark} flag, which is set to the Monte Carlo ID number corresponding to the chosen flavour. The generator consistently calls the appropriate analytic amplitudes, with the sole exception of the double real corrections, which are evaluated numerically using the \OpenLoops{} \texttt{bbhjj} library with a suitable flipping of the parton flavours. The number of active massless species is always set to \( n_f = 5 \), including in the running of the strong coupling and Yukawa factors. The theoretical uncertainty is estimated through the standard 7-point scale variation of the factorisation scale for the PDFs and the renormalisation scale with a correlated variation of Yukawa and strong coupling scales.

\subsection{Sensitivity of the Higgs $p_T$ spectrum to light-quark Yukawa couplings}
We begin the phenomenological analysis by focusing on the transverse-momentum spectrum of the Higgs boson. In~\fig{fig:lightpTHzoom}, we show the normalised distribution for the five different flavour channels. By normalising to the total cross section, we remove the dependence on the Yukawa coupling, making the comparison sensitive only to differences in the parton distribution functions (PDFs) across the channels. We observe that the position of the peak varies significantly with the initial-state flavour, with valence-quark channels exhibiting a softer spectrum. The comparatively harder spectrum in the \bbH{} channel is a consequence of the bottom-quark PDF being generated perturbatively, predominantly from gluon splitting.

Although the small masses of the first- and second-generation quarks make their contributions challenging to detect experimentally, the shape of the Higgs transverse-momentum distribution provides a potential handle to disentangle them from the dominant gluon-fusion channel. In particular, the softer spectrum associated with the charm-quark contribution, combined with the intermediate size of its Standard Model Yukawa coupling, makes this observable especially promising for setting experimental constraints in the $c\bar c \text{H}$ mode.

This shape behaviour for the Higgs transverse-momentum spectrum has also been observed with the analytic description of this observable at N$^3$LL$^{\prime}$+aN$^3$LO accuracy~\cite{Cal:2023mib} which is shown in the right panel of~\fig{fig:lightpTHzoom}.
We note that the spectra are normalised to the N$^3$LO cross sections obtained from \texttt{n3loxs}~\cite{Baglio:2022wzu} which uses the results from refs.~\cite{duhr:2019kwi,Duhr:2020kzd}\footnote{The code was modified to obtain the N$^3$LO cross sections for  $c\bar c \text{H}$ and  $s\bar s \text{H}$}.
At NNLL+NLO accuracy, the spectra of the different flavour channels already exhibit different shapes, which are confirmed at the more accurate \nnnres{} level with a substantial reduction of the scale uncertainty. The uncertainties for $b\bar b \text{H}$ are noticeably larger compared to the other channels.
In fact, the relative uncertainties for $b\bar b \text{H}$ at
a given order are of similar size as those for $s\bar s \text{H}$ at one lower order.
The main difference between the channels is the relative size of the PDF luminosities.
As already pointed out in~\citere{Cal:2023mib} for $b\bar b \text{H}$, the $b\bar b$ Born channel is numerically suppressed by the small b-quark PDFs, the
gluon-induced PDF channels which start at one higher order play a much more prominent role which explains the observed pattern of uncertainties for the different cases. Further, we want to note that the right panel of~\fig{fig:lightpTHzoom} is only shown for $p_T> 4\, \GeV$. With
the bottom mass $m_b = 4.18\, \GeV$ the assumption of $m_q \ll p_T$ and therefore our factorization
theorem are no longer valid for $p_T<4\, \GeV$.

\begin{figure}[t!]
\begin{center}
\begin{tabular}{cc}
	\includegraphics[width=.45\textwidth, page=1]{plots/5fs/light/ptHzoom_qqH.pdf}&
	\includegraphics[width=.45\textwidth, page=1]{plots/5fs/light/ptHzoom_qqHres.pdf}
\end{tabular}
\vspace*{1ex}
\caption{Transverse-momentum spectra of the Higgs boson obtained with \minnlo{} (left) and at N$^3$LL$^{\prime}$+aN$^3$LO, normalised to the total cross sections, for the different flavour-channel annihilation: bottom (blue, dashed), charm (green, solid), strange (pink, long-dashed), and down (brown, dotted), and  up (black, dotted-dashed) flavours.
\label{fig:lightpTHzoom}}
\end{center}
\end{figure}

\subsection{Simulations with diphoton signature in \minnlo{}}
In the POWHEG generator, the Higgs boson is treated as an on-shell asymptotic state. Exclusive simulations with colorless decay products can be performed by modeling its decay into two photons using the narrow-width approximation in \PYTHIA{8}, assuming a branching ratio of \({\rm BR}(H \to \gamma\gamma) = 0.227\%\)~\cite{LHCHiggsCrossSectionWorkingGroup:2016ypw}.
%In the \POWHEG{} generator, we consider the Higgs boson as an on-shell asymptotic state. We can perform exclusive simulations with colourless decay products by considering the decay into two photons is simulated using the narrow-width approximation in \PYTHIA{8}, assuming a branching fraction of \({\rm BR}(H \to \gamma\gamma) = 0.227\%\)~\cite{LHCHiggsCrossSectionWorkingGroup:2016ypw}. 
Inspired by a fiducial region accessible by both the ATLAS and CMS detectors we apply the following constraints on the rapidities and transverse momenta of the two photons:
\begin{equation}
|y(\gamma_i)|< 2.37, \quad
\frac{p_T(\gamma_1)}{m(\gamma_1, \gamma_2)} > 0.35,\quad \frac{p_T(\gamma_2)}{m(\gamma_1, \gamma_2)} > 0.25\,. \label{eq:aafidmycuts}
\end{equation}
Here, \( \gamma_1 \) denotes the hardest photon, i.e.\ the one with the largest transverse momentum. We apply the diphoton \texttt{fiducial cuts} defined in~\eqn{eq:aafidmycuts} and reconstruct the Higgs boson momentum through the components of the photon pair.

To enable cross-checks and comparisons with fixed-order predictions, we have extended the public code \SuSHi{}~\cite{Harlander:2012pb,Harlander:2003ai} to support calculations in light-quark parton fusion. In~\tab{tab:qqH_xs} we present the fully-integrated cross-section values for the total rate of Higgs boson production predicted by the extensions of \SuSHi{} and \minnlo{} according to the setup in~\sct{sec:setup}. We stress that the values are obtained using the bottom-quark Yukawa coupling and must therefore be rescaled by the ratio of the squared quark masses to obtain the correct magnitude of the Standard Model prediction. The central values show an enhancement of the cross section at fixed Yukawa coupling, driven by the parton luminosities. We observe good agreement between the fixed-order and \minnlo{} cross-section predictions, particularly in the down- and up-quark channels. Moreover, the scale uncertainties are significantly reduced in the down, up, and strange channels compared to those involving heavier flavours, most notably the bottom-quark fusion. In the last column of~\tab{tab:qqH_xs} we report the integrated cross-section numbers from the NNLO+PS simulation over the fiducial region defined by the diphoton cuts defined in~\eqn{eq:aafidmycuts}. We stress that the difference in the order of magnitude compared to the cross-sections for on-shell Higgs production is due to the Higgs branching ratio into photons. Among the initial-state quarks, the up quark shows the lowest efficiency in passing the fiducial selection, with only about 36\% of diphoton decay events surviving the cuts, compared to approximately 44\% for the down quark and 59\% for the charm quark.
\begin{table}[ht!]
  \vspace*{0.3ex}
  \begin{center}
	   \renewcommand{\arraystretch}{1.3}
    \begin{tabular}{|c||c|c||c|}
    \hline
    \makecell[c]{Flavour channel} & \makecell[c]{\shortstack{$\bar \sigma_{q\bar q \rightarrow \text{H}}$ (pb)\\ \SuSHi{}} } & \makecell[c]{\shortstack{$\bar \sigma_{q\bar q \rightarrow \text{H}}$ (pb)\\ \minnlo{}} }&  \makecell[c]{\shortstack{$\bar \sigma_{q\bar q \rightarrow \text{H}(\rightarrow\gamma \gamma)}$ (fb)\\ \texttt{fiducial $\gamma\gamma$ cuts}\\ \minnlo{}}}  \\
     \hline \hline
	    $d \bar d \rightarrow$ H & $11.46(6)_{-1.1\%}^{+0.5\%}$ & $11.442(3)_{-2.4\%}^{+2.8\%}$ & $11.420(5)_{-2.4\%}^{+2.8\%}$ \\
     \hline
	    $u \bar u \rightarrow$ H & $16.46(6)_{-1.1\%}^{+0.6\%}$ & $16.182(5)_{-2.3\%}^{+2.7\%}$  &  $13.169(8)_{-2.1\%}^{+2.5\%}$ \\
      \hline
	    $s \bar s \rightarrow$ H & $4.45(4)_{-1.4\%}^{+1.0\%}$ & $4.676(1)_{-1.8\%}^{+3.7\%}$  & $6.215(3)_{-1.7\%}^{+3.5\%}$  \\
       \hline
       $c \bar c \rightarrow$ H & $1.84(9)_{-2.9\%}^{+1.5\%}$ & $1.778(6)_{-0.9\%}^{+2.3\%}$  &  $2.399(1)_{-1.0\%}^{+2.3\%}$ \\
        \hline
        $b \bar b \rightarrow$ H &  $0.585(0)_{-9.2\%}^{+7.0\%}$ &  $0.5757(4)_{-8.0\%}^{+4.5\%}$ & $0.8089(8)_{-8.2\%}^{+4.7\%}$ \\
        \hline
    \end{tabular}
  \end{center}
  \vspace{-1em}
  \caption{
	 Comparison of \SuSHi{} cross-section numbers against the integrated \minnlo{} results for Higgs boson production via light-quark annihilation, with the Yukawa couplings set to the bottom-quark value. The last column presents the NNLO+PS results for $H\rightarrow \gamma\gamma$ production within the fiducial region defined in~\eqn{eq:aafidmycuts}. \label{tab:qqH_xs}}
\end{table}

\begin{figure}[h!]
\begin{center}
\begin{tabular}{cc}
\includegraphics[width=.45\textwidth, page=1]{plots/5fs/light/ptH.pdf}&
\includegraphics[width=.45\textwidth, page=1]{plots/5fs/light/pt_Higgs-aafid.pdf}
\end{tabular}
\vspace*{1ex}
\caption{Transverse momentum distribution of the reconstructed Higgs boson without cuts (left) and within the fiducial region defined in~\eqn{eq:aafidmycuts} (right).
The results are normalised to the prediction for down-quark fusion (brown, dotted) and compared to those for up- (black, dotted-dashed), strange- (pink, long-dashed), and charm-quark (green, solid) fusion.
\label{fig:lightpTH}}
\end{center}
\end{figure}

\begin{figure}[t!]
\begin{center}
\begin{tabular}{cc}
\includegraphics[width=.45\textwidth, page=1]{plots/5fs/light/yphoton1.pdf}&
\includegraphics[width=.45\textwidth, page=1]{plots/5fs/light/y_Higgs-aafid.pdf}
\end{tabular}
\vspace*{1ex}
\caption{Rapidity distribution of the hardest photon in the fully inclusive setup (left) and of the reconstructed Higgs boson in the fiducial region (right), for light-parton fusion initiated by up- (black, short-dashed), down- (brown, dotted), strange- (pink, long-dashed), and charm-quarks (blue, solid).\label{fig:lightrapidity}}
\end{center}
\end{figure}

We now present a selection of differential results. The shape of the Higgs transverse momentum spectrum for quark-initiated processes differs significantly from that in the gluon-initiated channel, offering a clearer possibility to study the interaction between the Higgs boson and partons. In the first plot of figure~\ref{fig:lightpTH}, we show the transverse momentum spectrum of the Higgs boson, reconstructed from the diphoton pair in the fully inclusive setup. It displays similar shapes across the different flavour channels, with normalisation reflecting the corresponding total cross sections. The second plot in figure~\ref{fig:lightpTH} shows the same distribution, but for events where the photons pass the fiducial cuts. In this scenario, more pronounced differences in shape emerge among the various channels. In particular, the up-quark contribution—about 1.5 times larger than the down-quark prediction in the fully inclusive case because of PDF enhancement—has a strong reduction in the fiducial region. Indeed, at intermediate values of transverse momentum, the up- and down-quark channels yield comparable results when the same Yukawa strength is assumed. Moreover, once the relative mass suppression in the Yukawa couplings is taken into account, the up-quark prediction falls below the down-quark one.

The main origin of this effect is the rapidity constraint defined in~\eqn{eq:aafidmycuts}. Indeed, the parton distribution functions influence the photon rapidity shape, yielding distinctive behaviour in the up-quark channel. The first plot in figure~\ref{fig:lightrapidity} shows the rapidity distribution of the photon with the highest transverse momentum across the different channels; similar patterns are observed for the second photon. For strange- and charm-quark fusion, as well as for the bottom-quark case, the distribution peaks around zero rapidity. In contrast, valence quark channels exhibit different shapes due to their characteristic parton luminosities. The down-quark channel shows a plateau for $|y_\gamma| < 2$, whereas the up-quark channel features a distribution not centred around zero rapidity, with a significant fraction of the cross section residing at larger rapidity values. As a result, the cut $|y_\gamma| < 2.37$ has a more pronounced impact on the up-quark distribution, removing a larger portion of the cross section when applying the fiducial selection. We have confirmed a similar trend at leading order, with a shape validated using the mass-rapidity scan of PDF4LHC luminosities obtained using \texttt{APFEL}~\cite{Bertone:2013vaa}. This distinctive behaviour, more evident here than in neutral Drell–Yan due to the higher Higgs mass, provides a useful handle for differential studies, as the rapidity spectrum in light-quark fusion differs markedly from dominant production modes. We conclude by presenting the Higgs boson rapidity, reconstructed from the two photons that satisfy the fiducial cuts~\eqref{eq:aafidmycuts}, as shown in the second plot of figure~\ref{fig:lightrapidity}. In the exclusive region, the up-quark behaviour is more similar to the down-quark, with a cross section between 1.1 and 1.3 times that of the down-quark process. The charm-quark and the strange-quark exhibit a steeper distribution compared to the Higgs boson from down-quark fusion.

In this section, we have presented the NNLO+PS generator for Higgs production via light-parton fusion with decay into photons as a potential tool to constrain light-quark Yukawa couplings. We conclude by noting that alternative channels sensitive to light-quark Yukawa couplings have been experimentally investigated recently, such as the Higgs decay rate into four leptons~\cite{CMS:2025xkn}.

\section{Conclusions}\label{sec:conclusions}
In the present paper, we have presented updated predictions for \bbH{} production at the LHC at a centre-of-mass energy of 13.6 TeV. All cross-sections are computed following the recommendations of the LHC Higgs Working Group. We first reported matched inclusive numbers obtained using the \nlonnllpart{} method with interpolations of the 13 TeV and 14 TeV cross-sections. The matching of the different flavour schemes has been performed only at the inclusive level for \bbH{} production, while differential matched predictions can be obtained using, for example, the methods presented in~\citeres{Gauld:2021zmq,guzzi:2024can}. In addition, we reported analytic predictions for the transverse momentum spectrum of the Higgs boson at $\text{N}^3\text{LL}^{\prime}+\text{aN}^3\text{LO}$. Matching with the exact $\text{N}^3\text{LO}$ prediction is highly desirable due to the importance of this observable in experimental studies of Higgs production via bottom-quark fusion.

We then presented the first numerical comparison of the \minnlo{} and \GENEVA{} generators for \bbtoH{} production in the massless scheme at NNLO+PS accuracy. They are compatible within theoretical uncertainties for the observables studied, showing good agreement with resummation results for the Higgs transverse momentum spectrum. We also included a proof-of-concept study of the \minnlo{} generator for BSM scenarios involving heavy-Higgs production in the enhanced bottom-quark channel. The NNLO corrections for the bottom-Yukawa contribution in the massive scheme have recently been computed with all-order parton shower matching in the \minnlo{} framework, with updated cross-sections compared against the massless prediction and lower-order results. The increasing QCD accuracy in this process shows improved compatibility between the two scheme choices and a corresponding reduction in theoretical uncertainty in the \bbH{} modelling.

\bbH{} production is an important background for HH searches in all channels involving at least one Higgs boson decaying into two bottom quarks. We discussed the impact of massive scheme predictions in fiducial phase-space regions relevant for HH studies. The top-quark Yukawa contribution is the dominant one, which can be modelled by selecting events from the \textsc{NNLOPS} ggF generator or using the MC@NLO generator for the $y_t^2$ terms at NLO+PS. We included a discussion on the impact of the bottom-Yukawa contribution, with a strong reduction in scale uncertainty using the \minnlo{} code in the 4FS. An important outlook for HH studies is the calculation of \bbH{} production via ggF Higgs boson production at NNLO+PS. Indeed, current \minnlo{} technology enables NNLO corrections for this process, with the main bottleneck being the missing two-loop contribution in the $y_t^2$ channel.

In the last section of the report, we discuss how to extend \bbH{} studies to Yukawa interactions with lighter quarks (\qqtoH{}). The shape of the transverse momentum spectrum is very sensitive to the different valence- and sea-quark PDFs that induce Higgs boson production. We presented a phenomenological study of \qqtoH{} production with the Higgs decaying into two photons, making use of \minnlo{} predictions in the massless scheme. To improve the theoretical input for constraints on the charm-quark Yukawa coupling in Higgs production, the calculation of NNLO(+PS) corrections for \ccH{} production in the 3FS is feasible with the \minnlo{} method by addressing numerical instabilities and using a well-justified massification of two-loop amplitudes, as performed for \bbH{} production in 4FS.

%In the present paper, we have presented updated predictions for \bbH{} production at the LHC at a centre-of-mass energy of 13.6 TeV. All the cross-sections are computed following the recommendations of the LHC Higgs Working Group. We firstly reported matched inclusive numbers obtained using the \nlonnllpart{} method with interpolations of 13 TeV and 14 TeV cross-sections. The matching of the different flavour schemes has been performed only at the inclusive level for \bbH{} production, while differential matched prediction can be performed using for examples the methods presented in~\citeres{Gauld:2021zmq,guzzi:2024can}. In addition, we have reported on the analytic predictions for the transverse momentum spectrum of the Higgs boson at $\text{N}^3\text{LL'+aN}^3\text{LO}$. The matching with exact $\text{N}^3\text{LO}$ prediction is highly desirable, due to the importance of this observable in the experimental studies for Higgs production via bottom-quark fusion. 
%We have then showed the first numerical comparison of the \minnlo{} and \GENEVA{} generators for the \bbtoH{} production in the massless scheme at NNLO+PS accuracy. They are compatible within the theoretical uncertainty for the studied observables, with a good compatibility with resummation results in the Higgs transverse momentum spectrum. We also include a proof-of-concept study of the \minnlo{} generator for BSM studies for heavy-Higgs production in the enhanced bottom-quark channel. The NNLO corrections for the bottom-Yukawa contribution in the massive scheme have been recently computed with an all-order parton shower matching in the \minnlo{} framework with updated cross-sections compared against the massless prediction and lower-order predictions. The increasing QCD accuracy in this process shows a better compatibility between the two scheme choices with an associated reduction of the theoretical uncertainty in the \bbH{} modelling.
%The \bbH{} production is an important background for HH searches in all the channels with at least one Higgs boson decaying into two bottom quarks. We discussed the impact of massive scheme predictions in the fiducial phase-space regions useful for HH studies. The top-quark Yukawa contribution is the dominant contribution, which can be modelled with a selection of events from the {\sc NNLOPS} ggF generator or usingthe MC@NLO generator for the $y_t^2$ terms at NLO+PS. We have included a discussion on the impact of the bottom-Yukawa contribution, with a strong reduction of the scale uncertainty using the \minnlo{} code in 4FS. An important outlook for HH studies is the calculation of the \bbH{} production via ggF Higgs boson production at NNLO+PS. Indeed, the current \minnlo{} technology enables NNLO corrections for this process with the main bottleneck provided by the missing two-loop contribution in the $y_t^2$ channel.
%In the last section of the report, we discuss how to extend the \bbH{} studies for Yukawa interactions with lighter quarks (\qqtoH{}). The shape of the transverse momentum spectrum is very sensitive to the different valence- and sea-quark PDFs that induce the Higgs boson production. We have presented a phenomenological study for \qqtoH{} production with the Higgs decaying into two photons, making use of novel \minnlo{} predictions in the massless scheme. In order to improve the theoretical input for constraints in the charm-quark Yukawa coupling in Higgs production, the calculation of NNLO(+PS) corrections for \ccH{} production in the 3FS is possible by addressing numerical instabilities and using a well-justified massification of two-loop amplitudes as performed in \bbH{} in 4FS.

\textbf{Citation policy. }The present work consists of contributions from different collaborations. If some of the results are employed for scientific publications, together with this work, one should cite the corresponding work in~\citeres{manzoni:2023qaf,Cal:2023mib,Biello:2024vdh,Biello:2024pgo,Gavardi:2025zpf}.

\textbf{Data Availability Statement.} All data used to produce the plots are available upon request by contacting the LHCHWG bbH/bH theory conveners via email.

\textbf{Acknowledgements. }This work has been carried out within the LHC Higgs Working Group, as a contribution to the Yellow Report 5. We thank all the WG3 conveners for coordinating the activities, in particular Tatjana Lenz. CB is grateful to Tommaso Giani for PDF insights and Jan Lukas Sp\"ah for interesting discussions on Yukawa interactions via light-quark fusion. We have used the Max Planck Computing and Data Facility (MPCDF) in Garching to carry out the \minnlo{} simulations presented here for \bbtoH{}, $q\bar q\rightarrow \text{H}$ and \bbH{} production. 
AG, RvK and FK have received funding from the European Research Council (ERC) under the European Union's Horizon 2020 research and innovation programme (Grant agreement 101002090 COLORFREE).
MZ~acknowledges the financial support by the MUR (Italy), with
funds of the European Union (NextGenerationEU), through the PRIN2022
grant 2022EZ3S3F.

\bibliography{bbh}
\bibliographystyle{JHEP}

\end{document}
